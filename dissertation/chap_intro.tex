\chapter{\bf Introduction}\label{chap:intro}
\epigraph{\textit{Bad men need nothing more to compass their ends, than that good men should look on and do nothing.}}{John Stuart Mill, \textit{inaugural address at St Andrews University, 1867.}}

\noindent The state of S\~{a}o Paulo is located in Brazil's Southeastern region and it is the most populous and developed (measured by the \textit{Índice de Desarrollo Humano} or IDH) region of the country.
It is an important state, accounting for 32\% of Brazil's Gross Domestic Product (GDP).
It has a high population of approximately 46.29 million (estimated for 2020) with a density of 166.25 inhabitants/km$^2$, and 29 million vehicles quantified for 2018 \citep[Brazilian Institute of Geography and Statistics,][]{IBGE2020}.  
The main urban area in this state is the Metropolitan Area of S\~{a}o Paulo (MASP), which contains 39 municipalities with a population of 21.9 million inhabitants \citep{IBGE2020b}, located around 770 m above sea level and around 45 km from the coast.
This megacity presents air pollution episodes associated with the exceedances of air quality standards for particulate matter (less than 10 $\mu$m (PM$_{10}$) and 2.5 $\mu$m (PM$_{2.5}$)) and ozone concentrations, which the vehicular emissions are the primary driver due to burning different types of fuels \citep{CETESB2019a}.
Also, short-lived climate pollutants (SLCP) such as tropospheric ozone and black carbon contribute significantly to climate change \citep{Von2015}.

Air quality is very susceptible to weather conditions, even if emissions are constant and the wind direction does not change \citep{Visscher2013}. 
Therefore, as climate change affects the weather conditions, it can impact air pollutant concentrations over time, worsening them, as is observed with surface ozone concentrations in the MASP \citep{CETESB2019}.
According to \citet{IPCC2013}, the potential effect of climate change on surface ozone in polluted regions suggests a `climate penalty' in which air emissions control will have to consider the temperature rise projection to achieve a specific target of ozone formation reduction.

According to World Health Organization (WHO) and recently studies \citep{Nuvolone2018}, people exposed to higher surface ozone concentrations, above WHO air quality guideline value (100 $\mu$g~m$^{-3}$ daily maximum rolling 8-hour mean), can have adverse health effects associated with respiratory diseases: trigger asthma, reduce lung function and cause lung diseases \citep{WHO2006}. 
Short-term exposure to higher ozone concentrations is associated with lung function impairment. 
It is identified as cough and pain on deep inspiration with significant variability for each individual depending on gender, age, and pre-existing pulmonary diseases \citep{Nuvolone2018}.
Tropospheric ozone is relevant because it is also a greenhouse gas, and there is robust evidence that it has a detrimental impact on vegetation physiology \citep{IPCC2013, Von2015}.

As part of the \href{http://www.metroclima.iag.usp.br/}{Metroclima project} (FAPESP 2016/18438-0), this study examines variations in the tropospheric (surface) ozone formation as a response to future changes in meteorological conditions in the MASP, without taking into account the future emission rate. 
The objective of the Metroclima project is to examine the role of the S\~{a}o Paulo megacity emissions as drivers for regional air quality degradation and climate change.
The project integrates multi-platform measurements and modeling tools to describe the atmospheric behavior (greenhouse gases and SLCP) and the effects of climate change on the air quality in the MASP.
Hence, based on the Metroclima project, this study analyzes two scenarios for the year 2030, where we take meteorological conditions into account based on the Radiative Concentration Pathways (RCP) concept for 4.5 and 8.5 W/m$^2$ as radiative forcing, named as RCP~4.5 and RCP~8.5, respectively \citep{VanVuuren2011a}.

This dissertation presents general background information in \textbf{Chapter~\ref{chap:intro}}, such as an overview of tropospheric ozone formation, climate change, previous studies related to this research for the MASP, motivations and the objectives of this study.
\textbf{Chapter~\ref{chap:metho}} presents a brief description of the methods related to the air quality model (setup, emissions inventory), meteorological data sets used for initial and lateral boundary conditions, static data (e.g., land use, land cover, surface terrain data), and the model performance evaluation.
\textbf{Chapter~\ref{chap:resul}} presents results about anthropogenic emissions, the model performance evaluation for air pollutants and weather parameters, and changes about contributions of meteorological projections under RCP scenarios to surface ozone in the MASP and surround it. 
Additionally,\textbf{Chapter~\ref{chap:resul}} presents explanations about ozone formation changes related to emission and meteorological factors.
Finally, in \textbf{Chapter~\ref{chap:concl}}, this dissertation shows conclusions that respond to the primary and specific objectives.
We also suggest future works based on the study limitations related to the period of analysis and the emission inventory projection based on local policy decisions about mitigation controls.

This study can support decision-makers because the ozone formation in the MASP can increase in the future, considering 2020 as one of the three warmest years on record globally according to the World Meteorological Organization \citep{WMO2020}. Consequently, in September 2020, the ozone formation reached higher concentrations, according to the CETESB report\footnote{Retrieved in this \href{https://cetesb.sp.gov.br/ar/wp-content/uploads/sites/28/2020/11/Boletim-Mensal-da-Qualidade-do-Ar-Setembro-2020.pdf}{link}.}.


%%%%%%%%%%%%%%%%
\section{Tropospheric ozone formation}\label{sec:ozone}
Tropospheric ozone is a secondary pollutant formed by reactions of nitrogen oxides (NO$_x$=NO+NO$_2$), volatile organic compounds (VOC), carbon monoxide (CO), and methane (CH$_4$) in the presence of solar radiation \citep{Von2015}.

Ozone is present everywhere. 
Rates of ozone formation are higher in polluted regions than in the remote\footnote{The definition of "remote" is somewhat ambiguous and should not be confused with "pristine". 
According to \citet{Wolfe2019}, remote troposphere means non land areas far away from forest and urban areas with significant air emission sources.} troposphere (i.e., oceans, according to \citealt{Wolfe2019}).
It depends on two major classes of precursors: VOCs (from anthropogenic and biogenic sources) and NO$_x$.
Ozone lifetimes vary in the troposphere depending on altitude, latitude, and season.
For example, during summer, the higher water vapor concentration and solar radiation reduce its lifetime \citep{Seinfeld2016}.

Tropospheric ozone photolysis is the principal source of hydroxyl radical (OH) in the presence of water vapor \citep{Brasseur1999}. 
This OH is often referred to as the `atmosphere detergent' because it defines the oxidizing capacity of the troposphere.
It reacts with most trace species (such as all organic compounds, CO, CH$_4$, and nitrogen and sulfur species) relevant to climate and air quality to produce carbon dioxide (CO$_2$) and water (H$_2$O).
Furthermore, the OH is the most important reactive species in the ozone formation because there is a competition between VOCs and NO$_x$ for the OH  \citep{Seinfeld2016}.

The basic photochemical cycle of NO$_x$ and O$_3$ occurs when the solar radiation ($\lambda$ $<$ 420 nm) dissociates NO$_2$ where atomic and excited oxygen (O$^3$P) is released as we can see in the following reaction \citep{Seinfeld2016}:
\begin{align}
\label{eq:no2_no}
&NO_2 + hv \rightarrow NO + O(^3P) & \lambda < 424 \ nm;
\intertext{then, this (O$^3$P) reacts with molecular oxygen (O$_2$) to produce ozone}
\label{eq:o_o3}
&O(^3P) + O_2 + M \rightarrow O_3 + M &;
\intertext{finally, nitrogen monoxide reacts with ozone to produce nitrogen dioxide and oxygen}
\label{eq:no_no2}
&NO + O_3 \rightarrow NO_2 + O_2. &
\end{align}

Where $M$ could be O$_2$ or N$_2$ that absorbs the excess energy. 
These reactions are known as the photostationary state that controls the ozone mixing ratio.
However, as \citet{Wallace2006} and \citet{Seinfeld2016} mention, there are other reactions with net ozone production rather than the photostationary state in the remote troposphere as well as regional and urban areas.
So, additional species (i.e., CO, CH$_4$, VOC) lead to the atmosphere's net ozone production.
The ozone formation is almost always initiated by reactions between a primary\footnote{For instance, methane (CH$_4$) and propane (C$_3$H$_8$) \citep{Sillman2014}.} hydrocarbon (RH), other organic or CO with the hydroxyl radical \citep{Sillman2014, Seinfeld2016}.
The reaction of the RH with OH radical removes hydrogen to produce a RO$_2$ radical, as shown in reaction (\ref{eq:rh_ro2}).
The CO oxidation exhibits many of the key reactions to analyze and understand the troposphere's chemistry.
In those reactions, it is relevant to consider the limits of low and high NO$_x$ concentrations.
The equivalent reaction for CO forms HO$_2$, a radical with many chemical similarities to the various RO$_2$ radicals \citep{Sillman2014}:

\begin{align}
\label{eq:rh_ro2}
    &RH + OH \xrightarrow[]{O_2} RO_2 + H_2O \\
    &CO + OH \xrightarrow[]{O_2} HO_2 + CO_2 
    \intertext{Other reactions with NO produce the conversion to NO$_2$,}
    &RO_2 + NO \xrightarrow[]{O_2} R'CHO + HO_2 + NO_2 \\
    \intertext{The R'CHO represents intermediate organic species or secondary VOC, typically including aldehydes and ketones.}
    &HO_2 + NO \rightarrow OH + NO_2
\end{align}

Then, photolysis of NO$_2$ results in the formation of atomic oxygen (O), which reacts with atmospheric O$_2$ to form ozone via reactions (\ref{eq:no2_no}) and (\ref{eq:o_o3}).
Also, by a reaction to itself (HO$_2$) and with nitrogen dioxide, highly soluble products are removed by wet deposition,

\begin{align}
    \label{eq:oh_remove}
    &2HO_2 \rightarrow H_2O_2 + O_2 \\ 
    &OH + NO_2+M \rightarrow HNO_3 + M. 
\end{align}

Ozone formation is highly dependent on sunlight.
Several authors and studies mention that the highest concentrations of ozone occur during the spring and summer periods due to high surface solar irradiation and temperature \citep{Von2015, Carvalho2015}. 
Urban areas (metropolitan and surrounding) present different chemical regimes when ozone is formed, referred to as NO$_x$-saturated (VOC-sensitive) or NO$_x$-sensitive (VOC-saturated).
These regimes are closely associated with their sources (produced by photolysis) and sinks of the odd hydrogen radicals (H, OH, HO$_2$, in general HO$_x$) \citep{Von2015}.

The VOC/NO$_x$ ratio is essential to understand how the ozone precursors are relevant to its formation or the increase/decrease behavior.
In areas with VOC/NO$_2$ ratio less than 5.5:1 predominates the OH-NO$_2$ reaction, retarding the further production of O$_3$; on the other hand, when the ratio exceeds 5.5:1, OH reacts with VOCs, accelerating O$_3$ production \citep{Seinfeld2016}. Other authors mention ratios between 8-12 between VOC and NO$_2$ \citep[for instant, ratio of 11 in the MASP, according to][]{Orlando2010}.
This analysis can be done using of the ozone isopleth plot\footnote{It is a helpful diagram to make the right decision about which pollutant emissions must be reduced \citep{Seinfeld2016}.} that shows the formation of ozone according to the VOC/NO$_x$ ratio.
Usually, many urban areas have a higher concentration of NO$_x$, called VOC-sensitive; on the other hand, there is a higher concentration of VOC in a rural area, called NO$_x$-sensitive \citep{Von2015}.

In urban areas, such as megacities (more than 10 million people according to World Urbanization Prospects\footnote{Retrieved in this \href{https://population.un.org/wup/Publications/Files/WUP2018-Report.pdf}{link}.} for 2018), different gaseous pollutants are released into the atmosphere, mainly due to vehicle emissions and industrial activities.
In South America, we have megacities with air quality problems such as Buenos Aires (Argentina), São Paulo (Brasil), Rio de Janeiro (Brasil), Lima (Peru), and Bogotá (Colombia). 
Figure~\ref{fig:o3_urban} illustrates complex photochemical reactions, when for one OH radical produced from one O$_3$, then two O$_3$ are provided from a complex mechanism of reactions.
According to \citet{Von2015}, many urban areas have higher NO$_x$ concentrations, which regime tends to be NO$_x$-saturated or VOC-sensitive.
Also, in urban areas is essential to consider a phenomenon called "NO$_x$ titration," in which, NO is the ozone sink via reaction (\ref{eq:no_no2}).

\begin{figure}
	\centering
    \includegraphics[width=.8\textwidth]{fig/o3_urban.pdf}
  	\caption{Tropospheric ozone formation in urban areas in the presence of VOC and NO$_x$.} {\scriptsize Note. Adapted from \citet{Jacob1999}. R is an organic group.}
  \label{fig:o3_urban}
\end{figure}

According to \citet{CETESB2019a} and \citet{Andrade2017}, vehicles in the MASP are responsible for the emissions of the majority of air pollutants.
Vehicles contribute to a high percentage of air pollutant emissions \citep{CETESB2019a}: 97\% of CO, 75\% of HC, 64\% of NO$_x$, 17\% of sulfur oxides (SO$_x$), and 40\% of particulate matter (PM).
There are different fuel types for the road transport sector in Brazil.
Heavy-duty vehicles use diesel-fueled which is a significant source of NO$_x$ emissions.
Flex-fuel vehicles can burn both gasoline C (around 75\% gasoline mixed with a range from 18\% to 27\% anhydrous ethanol, also called gasohol) or hydrous ethanol (7.5\% of maximum water content) \citep{CETESB2019a}.
Although ethanol in vehicles may lead to some reductions in CO and VOC emissions, it produces aldehydes during combustion mainly to form acetaldehyde in the exhaust emissions \citep{Gaffney2009}.
Primary acetaldehyde leads to O$_3$ formation, H$_2$O$_2$, formic acid, CO, peroxyacetyl nitrate (PAN), acetic acid, and peracetic acid \citep{Gaffney2009}.

In Brazil, unlike other countries, high concentrations of acetaldehyde have been found in the atmosphere \citep{Nogueira2014}, which in the MASP presents a NO$_x$-saturated or VOC-sensitive condition \citep{Sanchez-Ccoyllo2006, Alvim2018}.
In this context, VOC and CO have reactions with hydroxyl radicals that generate peroxy radicals. 
These (RO$_2$) compete with O$_3$ when they react with NO to produce NO$_2$, as shown in Figure~\ref{fig:o3_urban}.
\citet{Alvim2018} used the ozone isoplet Package for Research (OZIPR) trajectory model to determine the significant ozone precursors as VOC in S\~{a}o Paulo. 
They found that the ten most abundant VOC during the 2011-2012 period were ethanol, acetaldehyde, formaldehyde, acetone, propane, ethane, ethene, butane, 1-ethyl-4-methyl benzene, and 1,2,4-trimethylbenzene. 
Also, \citet{Alvim2018} presented an ozone isopleth by season, in which they illustrate one for spring based on the September 2011 period.
They found in the MASP VOC/NO$_x$ ratios less than 4, representing polluted urban areas with high concentrations ratio of NO$_x$ to VOCs; therefore, the VOC/NO$_x$ ratio is low, and ozone formation will depend on the VOC concentrations.
According to the same authors, the aldehydes (acetaldehyde and formaldehyde) were responsible for 74\% of the ozone formation, followed by aromatics (14.5\%). 
Therefore, simulation results in the MASP done by \citet{Alvim2018} showed that the most effective alternative for limiting the ozone formation is to reduce the VOC emissions through aldehydes from ethanol burning.
Those findings are in agreement with a recent study \citep{Dominutti2020}, in which their results suggest a strong influence of vehicular emissions in the VOC levels in the MASP, mainly associated with the large consumption of ethanol.

\section{Climate change and global warming}\label{sec:climate change}
Climate is an average meteorological condition over a long time, at least 30 years, according to the World Meteorological Organization (WMO) \citep{IPCC2013}. 
So, climate change is a variation of the normal meteorological conditions that persist over time.
Hence, weather and climate are not the same phenomenon.
Weather conditions always change every day and it is interesting for scientists to forecast extreme episodes using numerical weather models.

Climate change and global warming are different concepts but both are related.
Global warming depends on the greenhouse gases (GHG), however this last term does not mean something "bad".
The greenhouse effect is important to maintain the life on Earth where the GHG trap the heat, absorbing a fraction of the infrared (IR) waves emitted from the warm surface and is re-radiated by these GHG back to the Earth's surface \citep{Farmer2013}.
By definition, GHG are chemical species that: 
\begin{quote}
    "absorb and emit radiation at specific wavelengths within the spectrum of terrestrial radiation emitted by the Earth's surface, the atmosphere itself, and by clouds" \citep{IPCC2013}. 
\end{quote}
Water vapor (H$_2$O), carbon dioxide (CO$_2$), nitrous oxide (N$_2$O), methane (CH$_4$), and tropospheric ozone (O$_3$) are the primary GHG \citep{IPCC2013, Von2015}.

CO$_2$ is a very important GHG due to its capacity to warm the lower atmosphere efficiently, and it is known as the Earth's thermostat.
Life has evolved when CO$_2$ levels are approximately 280 ppm \citep{Farmer2013}, considering only natural forcing. 
Also, life has changed the composition of the Earth's atmosphere, governing the dynamics of CO$_2$ on the planet today \citep{Kasting1993}.

Since the Industrial Revolution, burning fossil fuels are the primary source of CO$_2$.
We can note in Figure~\ref{fig:co2_data} that the CO$_2$ levels have increased.
In 1960, it was 316 ppm with the rate of increase less than 1.0 ppm per year, and in 2020, it reached 417 ppm with the rate of increase of 2.4 ppm per year \citep{Tans2021, Letcher2021}.
Present-day atmosphere CO$_2$ levels exceed the natural equilibrium of absorption (oceans, biota, and land) with accumulated effect due to CO$_2$ has a prolonged life-time in the atmosphere because it is very unreactive \citep{Letcher2021}.
Unfortunately, this rising of CO$_2$ levels does not stop despite the warnings by scientists and institutions like the \citet{IPCC2013}.

\begin{figure}[htbp]
    \centering
    \includegraphics[width=8cm]{fig/co2_data_mlo.pdf}
    \includegraphics[width=8cm]{fig/co2_trend_mlo.pdf}
    \caption{Carbon dioxide concentrations in the atmosphere monitored at NOAA's Mauna Loa station, Hawaii observatory from 1958 to the present \citep{Tans2021}. \textit{With permission from \href{https://www.esrl.noaa.gov/gmd/ccgg/trends/mlo.html}{www.esrl.noaa.gov}}.}
    \label{fig:co2_data}
\end{figure}

We have experimented an increase of global mean temperature every year due to IR absorption from terrestrial radiation by rising CO$_2$ concentrations, re-radiated back toward the Earth's surface.
There can be more evaporation of water (e.g., oceans), increasing the vapor content in the troposphere, which more IR absorption \citep{Letcher2021}.
This warming effect by CO$_2$ is related to the climate sensitivity\footnote{\citet[Glossary Annex III]{IPCC2013}: "The effective climate sensitivity (units: $^\circ$C) is an estimate of the global mean surface temperature response to doubled carbon dioxide concentration that is evaluated from model output or observations for evolving non-equilibrium conditions. It is a measure of the strengths of the climate feedbacks at a particular time and may vary with forcing history and climate state, and therefore may differ from equilibrium climate sensitivity."} expressed by the feedback effect, whereby an increasing temperature causes an increasing concentration of water vapor in the atmosphere from oceans, which causes more increasing temperature with negative effects (permanent glacial melting, disappearing Arctic ice cap, change in cloud patterns, ocean acidification) \citep{Farmer2013, Letcher2021}.
Hence, this increase of global mean temperature is called global warming; moreover, \citet{Tuckett2021} mentions that this name has become \textbf{\textit{global heating}}.

\begin{figure}[htbp]
  \centering
  \includegraphics[width=13cm]{fig/FigSPM-05.pdf}
  \caption{Radiative forcing by emissions and drivers relative to 1750 up to 2011  \citep[ilustration adapted from][]{IPCC2013} }
  {\scriptsize Notes. Very height (VH), height (H), medium (M), low (L).}
  \label{fig:rf_emi}
\end{figure}

The interaction between gases and aerosols is very complex, some have negative and positive radiative forcing (RF)\footnote{According to \citet{IPCC2013},
"radiative forcing is the change in the net, downward minus upward, radiative flux (expressed in W~m$^{-2}$) at the tropopause or top of atmosphere due to a change in an external driver of climate change, such as, for example, a change in the concentration of carbon dioxide or the output of the Sun".
The IPCC report refers to RF as the change relative to the year 1750 as the global and annual average value.}.
However, there are estimates and associated uncertainties about those interactions related to radiative forcing, published by the \citet{IPCC2013}.
Figure~\ref{fig:rf_emi} illustrates the relationship between emitted compounds (i.e., gases and aerosols) and resulting atmospheric drivers considering the RF relative to the pre-industrial revolution (1750) up to 2011.
This chart shows that the cloud adjustments' level of confidence due to aerosols is low, which means that there are many uncertainties related to the cloud formation and the RF changes.
We again note that the tropospheric ozone is responsible for the positive RF more than the negative RF by decreasing the stratospheric ozone concentration. 

Considering all interactions, climate scientists demonstrated that human activities impacts the global climate due to the increase of GHG.
Figure \ref{fig:global} shows model results where we can see comparisons between two climate global simulations (i.e., only natural forcing and both natural and anthropogenic forcing) and observation data \citep{IPCC2013}.
We note again the model simulations that include the anthropogenic forcing are closer to the observations.
This behavior is a warning about how humankind will face the adverse effects in different environmental components (extended droughts, increasing wildfires, increasing urban air pollution, insect infestations, intensifies storms, changing rainfall, and agricultural patterns) \citep{Farmer2013}.

\begin{figure}[htb]
  \begin{center}
    \includegraphics[width=16cm]{fig/FigSPM-06.pdf}
  \end{center}
  \caption{Comparisons between two climate global simulations and observation data from 1850 up to 2010  \citep[ilustration adapted from][]{IPCC2013} }
  \label{fig:global}
\end{figure}

To understand future impacts due to atmospheric GHG and other pollutants, the \citet{IPCC2013} adopted four scenarios, using the concept of representative concentration pathways (RCP). 
According to the target level for 2100, the RCP emission scenario depends on the radiative forcing caused by GHG and other agents such as changes in land use and aerosol concentrations \citep{VanVuuren2011a}. 
Figure \ref{fig:RCPs} shows the effective radiative forcing trajectories, known as RCP 2.6, RCP 4.5, RCP 6.0, and RCP 8.5.
These scenarios include "time series of emissions and concentrations of GHG and aerosols and chemically active gases, as well as land use/land cover" as mentioned in \citep[Most et al., 2000; cited in][]{IPCC2013}.

In this dissertation, two scenarios (RCP 4.5 and RCP 8.5) were analyzed for 2030; in that year, those projections are close to others (RCP 2.6 and RCP 6.0), as shown in Figure~\ref{fig:RCPs}. 
After that year, uncertainties about land cover changes and anthropogenic emissions would increase significantly. 
The RCP 4.5 is a stabilization scenario and assumes all nations will comply with emission mitigation through changes in the energy system, including shifts to electricity from lower emissions energy technologies and carbon capture and geologic storage technology \citep{Thomson2011}.
The RCP 8.5 is the very high baseline emission scenario, representing the range of non-climate policy known as "business as usual," combined with the growing population and high demands of fossil fuel and food \citep{Riahi2011}.

\begin{figure}[htbp]
  \begin{center}
    \includegraphics[width=13cm]{fig/Fig8-22-1.pdf}
  \end{center}
  \caption{Effective radiative forcing for RCP scenarios \citep{IPCC2013}}{\scriptsize Notes:\\ WMGHG is Well-Mixed Green House Gases in dash-dot. Long dashes with squares are ozone; short dashes with diamonds are aerosol. RCPs 2.6, 4.5, and 6.0 net forcings at 2100 are approximate values using aerosol projected for RCP8.5, according to \cite{IPCC2013}.}
\label{fig:RCPs}
\end{figure}

\subsection{MASP and climate change}
According to several authors \citep{Andrade2017, Lima2018}, the MASP has a climate with mild temperatures, with a defined dry season (June to August) and humid summers (December to February).
There are rainfalls over the year (mainly during the summer season), influenced by weather systems such as cold fronts, the South Atlantic Convergence Zone, squall lines, sea breezes, and the urban effect \citep{Andrade2017, Lima2018}.
Three relevant factors dominate the air circulation \citep{Oliveira2003}: (i) sea breeze, (ii) mountain-valley circulation, and (iii) urban effects, such as roughness, building-barrier, and urban heat island (UHI) effects.
The sea breeze circulation influences the temperature differences within the MASP, producing a strong convergence zone during its lifetime (diurnal variation), which moves from southeast to northwest across the city \citep{Lima2018}.
The same authors also mentioned the UHI phenomenon's contribution to precipitation patterns changes due to convective air circulation when the synoptic-scale winds are weak.

The MASP has suffered from climate change events since the year 1930 \citep{Marengo2020}, with marked effects since 1960s \citep{Lima2018} due to the most remarkable growth of the urban spot and accelerated population increase due to the plentiful supply of jobs in the MASP during those years.
The main effects in the MASP due to climate change from 1960 until 2019 are: (i) increases in air temperature, extreme precipitation events, and atmospheric stability; (ii) reduction in the number of days with light precipitation, and relative humidity (mainly during the nights) \citep{Marengo2013, Lima2018, Nobre2019, Marengo2020}.
For the future, climate projection for Brazil showed trends of rising temperatures \citep{Nobre2019}.
Locally, the MASP may suffer the increase in the intensity and frequency of heavy precipitation and negative trends of light rain with the possibility of prolonged dry periods of continuous days \citep{Marengo2013}. 

\section{Previous studies for the MASP and motivation}\label{sec: prev studies}
Several modelling and observational studies about air quality in the MASP are available since 2006, mainly for fine particles and tropospheric ozone due to their impacts on human health and the environment.
\citet*{Sanchez-Ccoyllo2006} analyzed the tropospheric ozone formation in the MASP through the California Institute of Technology (CIT) model to propose emissions reduction to improve the air quality, based on VOC/NO$_x$ ratio.
They found that the urban area in the MASP has a VOC-sensitive condition, and the surrounding areas were slightly NO$_x$-sensitive.
Based on this information, they proposed that a reduction of VOC anthropogenic emissions could control ozone formation.

\citet{Vara2013} studied the impact of ozone formation in the MASP due to changes in its precursors' emission factors.
He mentioned the importance of the ozone precursors, being necessary to study with more relevance, the VOC.
This pollutant group encompasses several reactive compounds (e.g., isoprene, aromatics, aldehydes) that enhance ozone formation in urban areas.
Also, they concluded that the grid domain with 3 km of spatial resolution represented better the temporal ozone formation in the simulations with the WRF-Chem model.
Finally, regarding model parameterizations, they found that physical and chemical configurations in the WRF-Chem model were coherent to represent the ozone formation and its transport. 

Regarding the influence of climate change over ozone formation, \citet{Mazzoli2013} used the database from the global climatic model CCSM3 into the WRF-Chem model.
She analyzed two future years (2020 and 2050) compared with a case-control (the year 2011). 
She found minor differences in the ozone formation for the coming years when emissions do not change with time. 
However, she mentioned that if the air pollutant emissions increase, ozone formation will have a considerable impact on the MASP.

Several studies, for instance, \citet{Carvalho2015}, \citet{Andrade2015}, and \citet{Andrade2017} have been analyzing the air quality conditions related to emission sources, atmospheric chemistry, air quality modeling, and the evolution of pollutant concentrations due to regulations on air quality and emission sources.
\citet{Carvalho2015} highlighted that high ozone and particle concentrations are mostly associated with vehicular emissions, in which high ozone concentrations seasonal behavior occurs during the spring, probably due to lower cloud cover. % na media é verdade, mas aconteceu no 2014 maximas em verão.
They also presented recommendations for improving the air quality based on better public transport systems and economic incentives for clean-air technology.

\citet{Andrade2015} applied a novel bottom-up approach for road transport emission inventory based on the real conditions in the MASP, verified through the WRF-Chem model.
Simulations compared with measurements showed good quality for ozone; however, the approach requires improving the NO$_x$ and fine particles simulation results.
They mentioned uncertainties of emission inventories and the boundary conditions as aspects that have to be evaluated.
In that study, the operational model configurations for the gas-phase chemistry mechanism was the carbon-bond mechanism, version Z \citep[CBM-Z;][]{Zaveri1999}.
This gas-phase mechanism included ethanol and other oxygenated compounds to be represented explicitly, considering that the fuel consumption in the MASP has a high percentage of ethanol.

\citet{Andrade2017} emphasized the greatest challenge of controlling secondary pollutants such as ozone and fine particles.
They mentioned the ethanol biofuel impacts on the MASP as a relevant contributor of ozone formation in two ways: 
\begin{itemize}
	\item acetaldehyde emissions due to incomplete combustion,
	\item and the direct evaporative emission of ethanol.
\end{itemize}
Their review recommended improving the emission inventories for their use in the air quality modeling, including the stationary sources (use of wood and charcoal for cooking in restaurants, industrial process) and evaporative emissions (including gas stations). 
They suggested an effective way of improving air quality to scrap old vehicles (those 10-15 years of age) in the MASP according to recommended practices in most megacities.
Finally, they recommended studies based on the impact of climate change for different scenarios on air quality and human health.

\citet{Gavidia2018} studied how the chemical boundary conditions (CBC) impact the WRF-Chem model related to the tropospheric ozone formation in the MASP. 
He found the CBC did not considerably impact the spring period due to the increase of local sources and photochemical reactions during that time. 
By following \citet{Warner2011}, he used a grid-resolution (9 km x 9 km) with sufficient domain area to avoid propagation errors from meteorological boundary conditions related to the MASP, located at the center of the modeling domain.

Climate change projections and their assessment based on RCP scenarios for Brazil regions were developed until the year 2100 by \citet{Chou2014}, \citet{Cunningham2017}, \citet{Marengo2018}, and \citet{Nobre2019}.
As mentioned by \citet{Nobre2019}, Brazil was vulnerable to extreme climate events in the past (e.g., the year 2014).
For the future, climate projection for Brazil showed trends of rising temperatures \citep{Nobre2019}.
Consequently, as the term "climate penalty" effect suggests (mentioned in \citealt{IPCC2013}), tropospheric ozone concentrations could increase for future years. 
However, few studies \citep{Mazzoli2013,Schuch2020} analyzed the weather projections under climate change scenarios and their effects on the ozone formation in the MASP.

In recent work, \citet{Schuch2020} analyzed the sensitivity of the ozone concentration and fine particles (less than 2.5 $\mu$m) to change in emissions under the RCP~4.5 scenario over Brazil to determine the signal and spatial patterns, using short-period simulations.
Despite uncertainties about future changes in emissions and land use, they found a decrease in O$_3$ concentrations, located at the S\~{a}o Paulo and Rio de Janeiro metropolitan areas.
\citet{Schuch2020} mentioned a possible explanation that the difference is caused by the increase of NO$_x$ emissions in different VOC/NO$_x$ regimes.
However, in this study, the "climate penalty" effect due to temperature increase was not discussed related to its role in the tropospheric ozone formation, as suggested by the \citet{IPCC2013}.

Therefore, the primary motivation for carrying out this dissertation is to complement further analysis based on the climate penalty effect. 
Mainly associated with weather projections under two RCP scenarios that depict the intermediate and worst-case scenario, maintaining the same air pollutant emission rates for the evaluated period (2018) and the projected period (2030).
This dissertation's findings can be important for decision-makers and public policymakers about the ozone formation in urban and regional areas in the S\~{a}o Paulo state.
Also, this dissertation shows future works related to improving limitations found in this work, such as how the urban environment and land use will evolve in the following decades.
Furthermore, how can we add these land-use changes in the model configuration?


\section{Objectives}\label{sec:obj}
	
	This dissertation aims to study the impact of future climate change scenarios on tropospheric ozone formation in the Metropolitan Area of S\~{a}o Paulo (MASP) in 2030 using of the WRF-Chem model. Thus, specific objectives are:
	
	\begin{itemize}
		\item Prepare and update the emission files for the year 2018, representing the modeling domain areas centered in the S\~{a}o Paulo state.
		\item Setup the WRF-Chem model options for physical and chemistry modules to analyze ozone formation based on the MASP emission sources' features. 
		\item Evaluation of the WRF-Chem model results, based on the case-control study (September and October 2018).
		\item Obtain surface ozone concentrations from the WRF-Chem model based on the RCP scenarios and compared them with the model results representative of the case-control study.
		\item Study the current conditions that affect the tropospheric ozone formation through the WRF-Chem model compared with observations.
		\item Study how changes in future meteorological conditions in 2030 impact the surface ozone formation in the MASP and around it.
	\end{itemize}
	
