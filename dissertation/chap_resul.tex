\chapter{\bf Results and Discussion}\label{chap:resul}
\epigraph{\textit{Assure me that I yet may change these shadows you have shown me.}}{Charles Dickens, \textit{A Christmas Carol}}
   \noindent  This chapter presents the emission inventory results for September and October of 2018 and WRF-Chem model results for the second modeling domain ($\delta$X = 3 km) for all scenarios.
   Two scenarios (RCP 4.5 and RCP~8.5 for 2030) are future predictions oddly reminiscent of Dickens.
      
   In Section~\ref{sec:res_curr}, this study evaluates the WRF-Chem model results for the current conditions (September and October 2018 period) based on the meteorological IC/BC, emission for anthropogenic (road transport, industry, and residential), and biogenic sources.
   
   Statistical evaluation for surface ozone simulations covers detailed analyses for the global data, station types, and station locations.
   Further analyses show variations for some ozone precursors (NO$_x$, CO, and toluene) by hour of day as an average for the all analyzed period (i.e., Pinheiros station).
   
   In Section~\ref{sec:res_fut}, future meteorological conditions and surface ozone concentrations from the WRF-Chem model are presented. 
   These results are based on the same emission rates used to evaluate the model for current conditions, with the change for meteorological IC/BC provided from the CESM1 datasets: RCP 4.5 and RCP~8.5 scenarios; representing a set of possible humankind future pathways based on GHG emissions, air pollutants concentrations, and land use/land cover changes.   
   
   \section{Evaluation results for current conditions}\label{sec:res_curr}
   % First specific objective: prepare and update the emission file for the year 2018, representative of the modeling domain areas for the São Paulo state.
    
  \subsection{Anthropogenic emissions}\label{subsec:res_anth}
   This section shows the spatial and temporal anthropogenic emissions distribution.
   The results are based on emissions inventory approaches from the bottom-up methodology for the road transport sector \citep{Andrade2015} and from the EDGAR-HTAP datasets \citep{Janssens2015} interpolation in space and time using the anthro\_emiss emission pre-processor \citep{Kumar2020} for the industrial and residential sectors.
   
   We tested 11 experiments, running the model for seven\footnote{Two first days are spin-up, not included in the analysis.} non-rainy days based on different emission files, the base emission and some pollutants with a correction factor.
   Considering \citet{Emery2017} benchmarks, experiment 10 with a correction factor of 0.8 for NO$_x$ road transport emission files represented better the ozone formation in the MASP for September and October 2018.
   However, these tests showed the difficulty of improving other pollutants' simulations (NO$_x$, CO, and toluene) when compared with observations. 
   The emission files are the primary error source for simulations and require to be corrected as many works have reported for São Paulo and other cities outside of Brazil \citep{Russell2000, Holnicki2015, Andrade2017, Ibarra2020}.
   Vehicular emissions \citep[described in][]{Andrade2015} have limitations on the spatial and temporal distribution (e.g., roads only for heavy-duty vehicles), even when considering a bottom-up approach.
   
	Figure~\ref{fig:anth_dist}, as for instance, shows the spatial distribution of NO emissions for the second modeling domain.
	Based on CBM-Z chemical mechanism, 21  species of emission rates were considered in WRF-Chem emission files (i.e., \verb|wrfchemi_{00z, 12z}_{d01, d02}|).
	Figure \ref{fig:emiss_temp} shows emission rates and the temporal distribution of Industry, Residential, and Road Transport sectors. 
	Only road transport emissions present a temporal distribution based on information available in \citet{Andrade2015} (Figure~\ref{fig:temp_distr}).
	Table~\ref{tab:emi_d02} shows emission rates in units of kilo-tonnes per year (kt/year) accumulated in the second domain, based on the calculated emissions for the period September 2018.
	
	\begin{figure}[htb]
		\begin{center}
			\includegraphics[width=.9\textwidth]{fig/E_NO_emi_d02.pdf}
		\end{center}
  		\caption{NO emissions for second modeling domain in SE Brazil}
  		{\scriptsize Note:\\ Only Total emission files were used as input to the WRF-Chem model. Anthropogenic emissions were summed using a Python script \verb|wrfchemi.py|. LT= Local Time. X and Y axis labels are in  local coordinates.}
  		\label{fig:anth_dist}
	\end{figure}
	
	\begin{figure}[htpb]
		\centering
		\includegraphics[width=.8\textwidth]{fig/emi_kt.pdf}
		\includegraphics[width=.8\textwidth]{fig/emi_NO_time_kt.pdf}
  		\caption{Emission rates by species (part a) and temporal distribution (part b) for all days for NO emission rates and by sector}	
  		\label{fig:emiss_temp}
	\end{figure}
	
	\begin{table}
\centering
\caption{Emission rates (kt/year) for the second modeling domain based on activity data for the September 2018 period}
\label{tab:emi_d02}
\begin{tabular}{llrrr}
\toprule
{} &            ID &     Road &  Industrial &  Residential \\
Species                    &               &          &             &              \\
\midrule
Carbon monoxide            &            CO &  1412.29 &      510.65 &       191.80 \\
Nitrogen oxide             &            NO &   244.23 &      134.21 &         7.59 \\
Nitrogen dioxide           &        NO$_2$ &    27.14 &       22.87 &         1.29 \\
Nitrogen oxides            &        NO$_x$ &   271.37 &      157.08 &         8.88 \\
Sulfur dioxide             &        SO$_2$ &    30.74 &      133.09 &        12.92 \\
Ammonia                    &        NH$_3$ &     4.71 &       21.41 &         0.13 \\
Isoprene                   &           ISO &     0.52 &        0.00 &         0.00 \\
Ethane                     &           ETH &     8.31 &        6.53 &         0.73 \\
Propane                    &           HC3 &    52.41 &       39.19 &         4.39 \\
Alkanes (k$_{OH}$ 0.5 - 1) &           HC5 &    79.96 &       45.73 &         5.12 \\
Alkanes (k$_{OH}$ 1 - 2)   &           HC8 &   126.25 &       45.73 &         5.12 \\
Xylenes                    &           XYL &    55.87 &       19.60 &         2.19 \\
Alkenes (internal)         &           OL2 &    41.90 &       52.26 &         5.85 \\
Alkenes (terminal)         &           OLT &    75.85 &       45.73 &         5.12 \\
Alkenes (primary)          &           OLI &    48.41 &       26.13 &         2.92 \\
Toluene                    &           TOL &    80.50 &       26.13 &         2.92 \\
Formaldehyde               &          HCHO &    28.17 &       32.66 &         3.66 \\
Aldehydes                  &           ALD &    38.17 &       26.13 &         2.92 \\
Ketones                    &           KET &     0.45 &        0.00 &         0.00 \\
Methanol                   &      CH$_3$OH &     0.56 &        0.00 &         0.00 \\
Ethanol                    &  C$_2$H$_5$OH &   396.17 &      287.42 &        32.17 \\
\bottomrule
\end{tabular}
\end{table}




	
  \subsection{Meteorological model evaluation}\label{subsec:res_met}
  WRF-Chem model results were compared with measurements from CETESB stations network and the IAG/USP climatological station located in Água Funda.
The meteorological parameters extracted from the model for the statistical evaluation were 2-m temperature [$^\circ$C], 2-m relative humidity [\%], accumulated rain ($rainc$ and $rainnc$) [mm], wind speed [ms$^{-1}$] and direction [degree] at 10 m above the ground.

  Model simulations at IAG/USP location were analyzed. 
  This station complies with the World Meteorological Organization standards to locate a weather station \citep{WMO2018}.
  Furthermore, this station recorded the hourly rainfall rate, a parameter not recorded by the CETESB stations network.
  
  \subsubsection{IAG/USP climatological station}
Figure~\ref{fig:met_iag_comp} shows comparison between model simulations and observed values for each meteorological parameter, such as temperature, relative humidity, rain rate, wind speed, and wind direction.
Tables~\ref{tab:stats_iag} and \ref{tab: r_sign_iag} show statistical model evaluation results and t-test values for correlation coefficient for the IAG/USP station.

\begin{figure}[!hbt]
  \includegraphics[width=1\textwidth]{fig/met_IAG_comparison}
  \caption{Comparison between observed and modeled values for meteorological parameters for the IAG/USP station during the period Sep-Oct 2018}
  \label{fig:met_iag_comp}
\end{figure}

Temperature model results comply with MAGE and IOA statistical benchmarks suggested by \citet{Monk2019} for complex terrain.
However, MB result is not less than 1 ºC.
There is an overprediction of +1.56 ºC on average compared with observations.

Relative humidity from model results comply with the three statistical benchmarks (MAGE and MB units are in \%, apply for the MASP).
Thus, model results presented high accuracy to simulate this meteorological parameter.

Wind speed model results comply with the RMSE statistical benchmark, but not for the others (MB and IOA).
The model over-predicted in +1.69 m/s the average wind speed observations.
Wind direction model results comply with statistical benchmarks for MAGE (value less than 55) and MB (value less than 10).
Thus, the model has a good capability to simulate wind directions at IAG/USP station.
Figure~\ref{fig:wrplot_iag} shows the two wind rose plots, one with the modeled values and the other for the observation values. The frequency of calm winds is greater in observations than in the model results.

\begin{figure}[!hbt]
  \centering
  \includegraphics{fig/WRplot_IAG_sep_oct2018}
  \caption{Wind rose plot for Sep-Oct 2018 period at the IAG/USP station, for modeled and observed values. Wind speed simulations less than 0.5 m/s were replaced by 0 m/s to compare with observations.}
  \label{fig:wrplot_iag}
\end{figure}

Statistical results for rain simulations presented low performance, so the model was not accurate to represent rain observations recorded at the IAG/USP station.
However, the hypothesis t-test suggests that there is a significant linear relationship between modeled and observed values based on t-critical being less than the t-statistic value (Table~\ref{tab: r_sign_iag}).
Despite a low correlation, the rainfall simulation values have significant statistical indexes with the observations. 
Figure~\ref{fig:iag_daily_rain} shows comparisons between modeled and observed values for total daily rain with reasonable accuracy for September 14 and October 24, 2018.

\begin{figure}[!hbt]
	\centering
  \includegraphics[width=.8\textwidth]{fig/iag_daily_rain}
  \caption{Comparison between observed and modeled values for total daily rain rate during Sep-Oct 2018 for the IAG/USP station}
  \label{fig:iag_daily_rain}
\end{figure}

\begin{table}
\centering
\begin{threeparttable}
\caption{Statistical results for meteorological parameters for Sep-Oct 2018 (IAG/USP station)}
\label{tab:stats_iag}
\begin{tabular}{lrrrrr}
\toprule
Statistic\tnote{(a)} & 2-m Temp. ($^{\circ}$C) & 2-m RH (\%) & Rain rate (mm)  & W. Speed (m~s$^{-1}$) & W. Dir. ($^{\circ}$)  \\
\midrule
n      &  1461.00 &  1461.00 &  1461.00 &  1461.00 &  1461.00 \\
MB     &     1.56 &    -9.00 &     0.13 &     1.69 &     5.02 \\
MAGE   &     1.97 &    11.15 &     0.42 &     1.84 &    32.57 \\
RMSE   &     2.66 &    14.66 &     1.72 &     2.22 &        - \\
IOA    &     0.89 &     0.78 &     0.23 &     0.43 &        - \\
r      &     0.85 &     0.71 &     0.11 &     0.37 &        - \\
Mm     &    20.32 &    74.30 &     0.30 &     3.37 &        - \\
Om     &    18.76 &    83.30 &     0.18 &     1.69 &        - \\
Msd    &     3.96 &    15.68 &     1.20 &     1.50 &        - \\
Osd    &     3.89 &    14.46 &     1.35 &     0.91 &        - \\
t-stat &    61.63 &    38.51 &     4.23 &    15.21 &        - \\
t-crit &     1.96 &     1.96 &     1.96 &     1.96 &        - \\
\bottomrule
\end{tabular}
\begin{tablenotes}
{\scriptsize
	\item[(a)] MB = Mean bias, MAGE = Mean Absolute Gross Error, RMSE = Root Mean Square Error, Mm = Mean of modeled values, Om = Mean of observed values, Msd = Standard deviation of modeled values, and Osd = Standard deviation of observed values. Units depend on the meteorological parameter. Correlation coefficient (r) is in dimensionless units. Statistical parameters are t-test statistical (t-stat) and t critical (t-crit).}
\end{tablenotes}
\end{threeparttable}
\end{table}


\begin{table}
\centering
\caption{Correlation t-test values by meteorological parameter for Sep-Oct 2018 (IAG/USP station)}
\label{tab: r_sign_iag}
\begin{tabular}{lrrrr}
\toprule
{} &  2-m Temp. &   2-m RH &  W. Speed &  Rain rate \\
\midrule
t-statistic &  61.63 &  38.51 &  15.21 &  4.23 \\
t-critical  &   1.96 &   1.96 &   1.96 &  1.96 \\
\bottomrule
\end{tabular}
\end{table}



\subsubsection{CETESB and IAG/USP grouped by station types}
Statistical results considered all stations with non missing values for temperature, relative humidity, wind speed, and wind direction.
Table~\ref{tab:stats_all} shows global statistical results for each meteorological parameter.

\begin{table}[ht]
\centering
\caption{Statistical results for meteorological parameters for Sep-Oct 2018 (all stations)}
\label{tab:stats_all}
\begin{tabular}{lrrrrrrrrrr}
\toprule
{} &      n &     MB &   MAGE &   RMSE &   IOA &     r &     Mm &     Om &    Msd &    Osd \\
\midrule
2-m Temp. ($^{\circ}$C) 	&  38727 &   1.30 &   1.94 &   2.54 &  0.92 &  0.89 &  21.99 &  20.69 &   4.36 &   4.74 \\
2-m RH (\%)				&  37298 &  -7.32 &  11.18 &  14.46 &  0.84 &  0.76 &  68.87 &  76.19 &  17.71 &  18.49 \\
W. Speed (m~s$^{-1}$) 	&  45412 &   1.38 &   1.71 &   2.13 &  0.52 &  0.39 &   3.36 &   1.98 &   1.70 &   1.12 \\
W. Dir. ($^{\circ}$) 	&  43189 & -21.66 &  50.58 &      - &     - &     - &      - &      - &      - &      - \\
\bottomrule

\end{tabular}
\end{table}


Model results for temperature comply with two statistical benchmark (MAGE and IOA).
The model over-predicted temperature results with +1.31 ºC based on modeled (Mm) and observed (Om) mean values, also shown as MB result in Table~\ref{tab:stats_all}.

In general terms, the model has a good performance for relative humidity because simulations comply with all the benchmarks (MAGE < 20\%, MB < $\pm$ 10\%, and IOA $\geq$ 0.6).
However, simulations under-predicted measurements as we can see in the negative MB value.

Model simulations for wind speed comply with two (RMSE and MB) of three statistical benchmarks.
The IOA value does not reach the statistical benchmark ($\geq$ 0.6).
Regarding to wind direction, only one statistical value (MAGE) complies with the benchmark ($\leq$ 55).
The MB (as absolute value) for wind direction is greater than the statistical benchmark ($\leq~$10).

Considering correlation results, Table~\ref{tab: r_sign_met_all} shows t-test values and number of samplers (n) analyzed; which indicate a significant linear relationship between modeled and observed values due to t-statistic values are greater than t-critical values.

\begin{table}
\centering
\caption{Correlation t-test values by meteorological parameter for Sep-Oct 2018 (all stations)}
\label{tab: r_sign_met_all}
\begin{tabular}{lrrr}
\toprule
{} &     2-m Temp. (ºC) &  2-m RH (\%) &  W. Speed (m~s$^{-1}$) \\
\midrule
t-statistic &    390.21 &     236.90 &     94.53 \\
t-critical  &      1.96 &       1.96 &      1.96 \\
n           &  39966.00 &   38538.00 &  46920.00 \\
\bottomrule
\end{tabular}
\end{table}



Further specific analysis by station type are shown in Appendix Table~\ref{tab:stats_all_type}, and summarized in Table~\ref{tab:sum_bench}, which shows model performance results for each meteorological parameter and station type.
The WRF-Chem model has a good performance for the relative humidity simulations.
Figure~\ref{fig:rh_cetesb} in Appendix shows time series as a daily mean for relative humidity, where model results and observations are compared by station type. 
Temperature model results comply with two of three statistical benchmarks.
Figure~\ref{fig:temp_cetesb} in Appendix shows temperature values as a daily mean for model results compared with observations by each station type.
Only model results for "Industry station type" presented good performance.

Finally, the wind was difficult to simulate and did not present good results compared with observed values.
Statistical results suggest low capability of the model to simulate wind speed in urban and forest areas (i.e., Pico do Jaraguá station).
However, this is not particularly surprising given that CETESB stations aim to measure air quality parameters, and many stations could not comply with WMO recommendations to install weather station \citep{WMO2018}.
Furthermore, there are errors in the model due to low resolution to represent the topography, such as hills in the higher terrain (i.e., Pico do Jaraguá). 
Figure~\ref{fig:ws_cetesb} in Appendix shows wind speed values as a daily mean for model results and observations; modeling values belonging to "Forest preservation sites" highly overpredicted observed values more than other station types.

\begin{table}
\centering
\caption{Summary of compliance of statistical benchmarks }
\label{tab:sum_bench}
\begin{tabular}{lrrrrrr}
\toprule
{}        &    Urban &  U. park &  R. urban &    Ind. &  F. pre. & \\
\midrule
2-m Temp. ($^{\circ}$C)  (3 benchmarks)   &    $\checkmark$$\checkmark$ &  $\checkmark$$\checkmark$ &    $\checkmark$$\checkmark$ &  $\checkmark$$\checkmark$$\checkmark$ &    $\checkmark$$\checkmark$ & 11 \\
2-m RH (3 benchmarks)  &     $\checkmark$$\checkmark$$\checkmark$ &   $\checkmark$$\checkmark$ &     $\checkmark$$\checkmark$$\checkmark$ &   $\checkmark$$\checkmark$$\checkmark$ &     $\checkmark$$\checkmark$$\checkmark$ & 14\\
W. Speed (m s$^{-1}$) (3 benchmarks) &     $\checkmark$$\checkmark$ &   $\checkmark$ &     $\checkmark$$\checkmark$ &   $\checkmark$ &   & 6  \\
W. Dir. ($^{\circ}$) (2 benchmarks)  &     $\checkmark$ &   $\checkmark$$\checkmark$ &      &   $\checkmark$ &     $\checkmark$ & 5\\
\bottomrule
 & 8 & 7 & 7 & 8 & 6\\
\end{tabular}
\end{table}
	
  \subsection{Air quality model evaluation} \label{subsec:res_aq}
  This section shows the model output for current conditions (Sep. and Oct. 2018), their evaluation through statistical analysis, and the comparison between model simulations and observations from CETESB air quality stations network.
  The parameters extracted from the model for the statistical evaluation were nitrogen monoxide (NO in $\mu$g~m$^{-3}$), nitrogen dioxide (NO$_2$ in $\mu$g~m$^{-3}$), carbon monoxide (CO in ppm), toluene ($\mu$g~m$^{-3}$), and surface ozone (O$_3$ in $\mu$g~m$^{-3}$).
   Ozone 8-hr rolling mean was also calculated from hourly time series for model simulated and observed values.
   Some stations located close to coastal zone were not included as part of statistical evaluation due to low performance of WRF-Chem model, which are:
   
   \begin{itemize}
   	\item Santos
   	\item Santos-Ponta da Praia
   	\item Cubat\~{a}o-Centro
   	\item Cubat\~{a}o-Vale do Mogi
   	\item Cubat\~{a}o-V. Parisi
   \end{itemize}
   
   Furthermore, model hourly simulations for September 14-15, 2018, were not considered in the statistical analysis due to cloud and rainfall conditions not represented appropriately by the WRF-Chem model.
   Figure~\ref{fig:rain_sep18} in Appendix shows mean cloud cover and total rain by day.
   Thus, 57 stations with hourly observations were compared with WRF-Chem model results, according to the station types (i.e., Forest Preservation, Urban, Urban Park, Regional Urban, and Industry).
   As highlighted in Figure~\ref{fig:mapStations}, the station types inside the MASP correspond to Forest Preservation, Urban, and Urban Park.

  Global statistical results in Table~\ref{tab: gl_st} show correlation coefficients (r) for ozone  above 0.50 that comply with the criteria level as statistical benchmark suggested by \citet{Emery2017}.
  Average values of NMB and NME for ozone comply with the goal level ($\leq\pm$5 \% for NMB) and criteria level ($\leq$25 \% for NME).
  Based on MB results, positive values are indicators that simulations are overestimating the observations.
  
  Primary pollutants (NO$_x$ and CO) have low values of correlation coefficients.
  On average, the model over-predicted NO$_2$ concentrations due to positive values for MB (+4.12 $\mu$g m$^{-3}$ for Sep. 2018) and NMB (15.69~\% for Sep. 2018).
  However, simulations for September 2018 period under-predicted NO and CO concentrations which NMB values are -6.39~\% and -52.3~\%, respectively.
  These underpredictions increase for October 2018 period with -48.06~\% and -61.77~\%, respectively.
  Probably, NO and CO emissions could be underestimated, mainly for the October 2018 period.
  The spatial distribution of NO$_x$ emission is subjected to errors, considering that heavy-duty vehicles in the MASP use specific roads where trucks are concentrated on motorways around the city \citep{Ibarra2020}.
  
  \begin{table}
\centering
\caption{Global statistical results for air quality parameter}
\label{tab: gl_st}
\begin{tabular}{lrrrrrrrrrr}
\toprule
{} & \multicolumn{5}{l}{Sep. 2018} & \multicolumn{5}{l}{Oct. 2018} \\
{} &        o3 &        no &       no2 &       nox &       co &        o3 &        no &       no2 &       nox &        co \\
\midrule
n     &  23921.00 &  21693.00 &  21693.00 &  21693.00 &  9837.00 &  26034.00 &  24099.00 &  24099.00 &  24099.00 &  12152.00 \\
MB    &      9.32 &     -0.55 &      4.09 &      3.54 &    -0.26 &     13.15 &     -4.10 &      1.39 &     -2.71 &     -0.31 \\
MAGE  &     24.01 &     10.50 &     18.33 &     27.75 &     0.29 &     22.51 &      8.73 &     15.44 &     22.99 &      0.33 \\
RMSE  &     30.07 &     23.78 &     25.77 &     43.63 &     0.40 &     29.43 &     21.89 &     21.79 &     37.44 &      0.44 \\
NMB   &     19.24 &     -6.54 &     15.56 &     10.20 &   -52.28 &     31.65 &    -47.84 &      5.92 &     -8.46 &    -61.76 \\
NME   &     49.57 &    124.67 &     69.75 &     79.99 &    59.26 &     54.17 &    101.81 &     65.74 &     71.70 &     64.64 \\
IOA   &      0.80 &      0.46 &      0.62 &      0.58 &     0.45 &      0.76 &      0.31 &      0.65 &      0.56 &      0.44 \\
r     &      0.67 &      0.25 &      0.39 &      0.34 &     0.19 &      0.64 &      0.16 &      0.42 &      0.34 &      0.17 \\
Mm    &     57.75 &      7.87 &     30.36 &     38.23 &     0.24 &     54.71 &      4.47 &     24.87 &     29.35 &      0.19 \\
Om    &     48.43 &      8.42 &     26.27 &     34.70 &     0.49 &     41.56 &      8.58 &     23.48 &     32.06 &      0.51 \\
Msd   &     37.75 &     17.50 &     25.47 &     38.68 &     0.12 &     33.27 &      8.57 &     20.81 &     26.71 &      0.08 \\
Osd   &     31.36 &     20.97 &     19.99 &     36.90 &     0.31 &     27.49 &     21.14 &     19.47 &     36.83 &      0.32 \\
NMB 2 &      2.24 &       NaN &       NaN &       NaN &      NaN &      1.92 &       NaN &       NaN &       NaN &       NaN \\
NME 2 &     21.66 &       NaN &       NaN &       NaN &      NaN &     20.80 &       NaN &       NaN &       NaN &       NaN \\
\bottomrule
\end{tabular}
\end{table}


  
   \subsubsection{Surface ozone}
  Ozone model results were also evaluated for each station type, as shown in Table~\ref{tab: o3_sta}. 
  Three statistical values (NMB, NME, and r) for many station types complied with at least the criteria level of benchmarks suggested by \citet{Emery2017}.
  NMB values for ozone are less than the range $\pm$15\% for those stations in the MASP.
  Only Forest preservation (F. pre.) and Urban park (U. park) stations comply with some statistical goal level of benchmarks, such as NMB (values <~$\pm$5\%) for September 2018 period.
  Only for stations classified as Industry (e.g., Santa Gertrudes), the correlation coefficient value for September 2018 complies with the goal level as a statistical benchmark (r >~0.75).
  Based on hypothesis t-test evaluation, there is a significant linear relationship between observed and simulated values because the correlation coefficient is significantly different from zero, as shown in Table~\ref{tab: r_sign} ($\alpha$ = 0.05 at confidence interval of 95~\%).
  We can see that t-statistic values are greater than t critical values, which mean P-values are sufficiently small to reject the null hypothesis and accept the alternative hypothesis: "there is enough evidence to reject the null hypothesis (H$_o$)".
  \begin{table}
\begin{threeparttable}[b]
\centering
\caption{Statistical results for surface ozone by station type}
\label{tab: o3_sta}
\begin{tabular}{lrrrrrrrrrr}
\toprule
Month & \multicolumn{5}{l}{Sep. 2018} & \multicolumn{5}{l}{Oct. 2018} \\
type &   F. pre. &       Urb &  U. park &     Ind &  R. urb. &   F. pre. &       Urb &  U. park &     Ind &  R. urb. \\
Statistic\tnote{(a)} \\
\midrule
n    &    636.00 &  11314.00 &  2205.00 &  637.00 &  9057.00 &    709.00 &  12383.00 &  2663.00 &  706.00 &  9500.00 \\
MB   &      0.51 &      4.03 &    -1.68 &   17.20 &    18.75 &      4.46 &     10.33 &     4.55 &   19.91 &    19.50 \\
MAGE &     24.44 &     23.27 &    23.35 &   27.23 &    24.83 &     22.11 &     21.25 &    21.03 &   28.69 &    24.11 \\
RMSE &     30.64 &     29.54 &    29.25 &   32.52 &    30.70 &     28.38 &     28.46 &    27.69 &   35.10 &    30.71 \\
NMB\tnote{(b)}  &      \textcolor{blue}{\bf 1.02} &      \textcolor{blue}{8.96} &    \textcolor{blue}{\bf -3.49} &   \textcolor{red}{33.95} &    \textcolor{red}{35.57} &     \textcolor{blue}{10.75} &     \textcolor{red}{27.77} &    \textcolor{blue}{11.52} &   \textcolor{red}{45.85} &    \textcolor{red}{40.85} \\
NME\tnote{(b)}  &     \textcolor{red}{48.90} &     \textcolor{red}{51.75} &    \textcolor{red}{48.54} &   \textcolor{red}{53.75} &    \textcolor{red}{47.11} &     \textcolor{red}{53.28} &     \textcolor{red}{57.12} &    \textcolor{red}{53.21} &   \textcolor{red}{66.05} &    \textcolor{red}{50.50} \\
\bf NMB\tnote{(c)} & \textcolor{blue}{7.19} & \textcolor{blue}{7.39} & \textcolor{blue}{\bf -0.58} & \textcolor{blue}{-7.26} &    \textcolor{blue}{\bf -0.28} &     \textcolor{blue}{\bf -4.48} &  \textcolor{blue}{8.97} &     \textcolor{blue}{\bf 4.75} &   \textcolor{blue}{-8.53} &    \textcolor{blue}{\bf -1.87} \\
\bf NME\tnote{(c)} & \textcolor{red}{30.59} &     \textcolor{blue}{24.75} &    \textcolor{blue}{22.67} &   \textcolor{blue}{17.46} &    \textcolor{blue}{19.22} &     \textcolor{blue}{24.36} &     \textcolor{blue}{23.26} &    \textcolor{blue}{23.02} &   \textcolor{blue}{20.63} &    \textcolor{blue}{18.57} \\
IOA  &      0.81 &      0.80 &     0.82 &    0.85 &     0.76 &      0.74 &      0.77 &     0.80 &    0.76 &     0.69 \\
\bf r   &      \textcolor{blue}{0.72} &      \textcolor{blue}{0.68} &     \textcolor{blue}{0.69} &    \textcolor{blue}{\bf 0.80} &     \textcolor{blue}{0.69} &      \textcolor{blue}{0.60} &      \textcolor{blue}{0.66} &     \textcolor{blue}{0.67} &    \textcolor{blue}{0.69} &     \textcolor{blue}{0.60} \\
Mm   &     50.49 &     49.00 &    46.43 &   67.85 &    71.45 &     45.97 &     47.52 &    44.07 &   63.34 &    67.25 \\
Om   &     49.98 &     44.97 &    48.11 &   50.65 &    52.70 &     41.51 &     37.20 &    39.52 &   43.43 &    47.74 \\
Msd  &     43.91 &     39.91 &    40.11 &   36.60 &    28.55 &     34.73 &     34.90 &    36.53 &   30.54 &    25.17 \\
Osd  &     28.98 &     28.95 &    31.62 &   46.19 &    32.56 &     24.68 &     25.90 &    27.90 &   39.49 &    27.37 \\
\bottomrule
\end{tabular}
\begin{tablenotes}
{\scriptsize
	\item[(a)] MB, MAGE, RMSE, Mm, Om, Msd, and Osd values are in $\mu$g~m$^{-3}$. Correlation coefficient is in dimensionless units.
	\item[(b)] No cutoff was applied.
	\item[(c)] A cutoff value of 80 $\mu$g~m$^{-3}$ was applied to calculate NMB and NME only for 1-hr ozone, suggested by \citet{Emery2017}. Units are in percentage. Units are in percentage. Values in bold blue comply with the Goal benchmark and in only blue with the Criteria benchmark for ozone. Values in red don't comply with the statistical benchmarks.}
\end{tablenotes}
\end{threeparttable}
\end{table}


  \begin{table}
\centering
\caption{t-test values for correlation coefficient (r) for Sep. 2018}
\label{tab: r_sign}
\begin{tabular}{lrrrrr}
\toprule
{} &  F. pre. &    Urb &  U. park &    Ind &  R. urb. \\
\midrule
t-statistic &    26.21 &  98.94 &    44.87 &  33.70 &    91.00 \\
t critical  &     1.96 &   1.96 &     1.96 &   1.96 &     1.96 \\
\bottomrule
\end{tabular}
\end{table}
  
 Based on the score shown in \ref{tab:bench_o3}, both months presented a better performance of simulations for ozone. 
Stations belonging to the MASP comply with two of three statistical benchmarks, as shown in Table \ref{tab:bench_o3}.
 Despite of cloudy and rain conditions, simulations for October 2018 presented good performance as September 2018.
  Likewise, the cold front system reached the MASP for six days during October 5-10, 2018, according to the \textit{Centro de Hidrografia da Marinha} \citep{CHM2020}.
  As was mentioned at the beginning of this chapter, the WRF-Chem model, due to its configuration, could not simulate adequately heavy rainy days and consequently their influence in the ozone formation.
  Furthermore, the model performance evaluation (Table \ref{tab:bench_o3}) revealed that surface ozone simulations presented better values for station types as `Forest preservation' (only for October), `Urban park', `Industry' (only for September), and `Regional urban'.
  A possible explanation for the low performance attributed to Urban stations could be related to the model resolution, in which spatial emission rates distribution (3 km $\times$ 3 km) do not represent local emission for stations closer to main roads.
  
\begin{table}[b]
\centering
\caption{Summary of compliance of the ozone statistical benchmarks by station types}
\label{tab:bench_o3}
\begin{tabular}{lrrrrrrr}
\toprule
Month & Statistic        &   F. pre.	 &  Urban     &  U. park &   Ind   &  R. urb. & \\
\midrule
Sep. 2018 & NMB 		 &  \ok	     &  \ok       &  \ok \ok & \ok     & \ok \ok  & 7  \\
          & NME          &           &  \ok       &  \ok     & \ok     & \ok      & 4  \\
          & r            &  \ok      &  \ok       &  \ok     & \ok \ok & \ok      & 6  \\
          & \bf  Total   &  2		 & 3 		  & 4 		 & 4 	   & 4        & \bf 17 \\
          {} \\
Oct. 2018 & NMB 		 & \ok \ok   & \ok		  & \ok \ok  & \ok 	   & \ok \ok  & 8 \\
          & NME    		 & \ok		 & \ok		  & \ok		 & \ok	   & \ok      & 5 \\
          & r 			 & \ok		 & \ok        & \ok      & \ok     & \ok      & 5 \\
          & \bf Total    & 4 		 & 3 		  & 4 		 & 3		   & 3        & \bf{17} \\
\bottomrule
\multicolumn{8}{l}{\scriptsize Note. \ok \ok = goal and criteria levels compliance suggested by \citet{Emery2017}.  }\\
\end{tabular}
\end{table}

  If we can see details, Figure~\ref{fig: o3_stats} shows statistical evaluations for each station.
 Correlation (\textit{r}) values comply with the criteria level in all stations, and few of them comply with the statistical goal level.
  However, in some locations (8 stations), the model results do not comply with the criteria level for NMB.
  The statistical evaluation also calculated the maximum model overprediction value (NMB equals to 41.40\% in September 2018), located in Sorocaba station, belonging to the "Regional urban" station type.
 The model simulations only in twelve stations on September 2018 underestimated ozone concentrations between -19.17\% (Araraquara, as an "Regional urban" station type) and -2.02\% (Piracicaba, as an "Regional urban" station type) as NMB values.
  
\begin{figure}[!ht]
\begin{center}
    \includegraphics[width=1\textwidth]{fig/o3_stats.pdf}
\end{center}
  \caption{Statistical results for surface ozone during the period of September 2018}
  \label{fig: o3_stats}
\end{figure}

  Finally, model simulations of ozone for many stations (25) on September 2018 complied with the criteria level for the NME statistical benchmark suggested by \citet{Emery2017}. Only three `Regional urban' stations complied with the goal level, which are Limeira, Guaratinguet\'{a}, and Taubat\'{e}.
  IOA results are also shown in Figure~\ref{fig: o3_stats} with very high values above 0.6.
  According to \citet{Willmott1984}, "the index of agreement varies between 0 and 1 where a value of 1 expresses perfect agreement between \textit{O} (observations) and \textit{P} (predictions) and 0 describes complete disagreement."
  IOA values greater than 0.8 suggest that the model is highly accurate for those stations that comply with that criteria (e.g., Paulínia, Campinas Taquaral, Piracicaba, Americana, Jundiaí, Limeira, Carapicuíba, Santana, Pico do Jaraguá, Cid. Universitária USP IPEN, Ibirapuera, São Caetano do Sul, Diadema, Guarulhos-Paço Municipal, Parque D. Pedro II, Interlagos, and Itaquera).

Figure~\ref{fig:Sep18_type_o3} shows the comparison in time between observed (black dots) and modeled (green line) surface ozone values for September 2018 period.
September 14-15 were not considered due to the high inaccuracy of the WRF-Chem model to simulate cloud.   
Ozone concentrations as daily maximum 8-h rolling mean (MDA8) were also calculated for observed and simulated values, shown in Figure~\ref{fig: MDA8_type_current}.
The are many stations to plot model simulations compared with observations.
For that reason, comparison values are classified by station type.
Both figures show peak values for model simulations and observations.
This consideration is essential for model validation where model capabilities reproduce peak ozone concentrations \citep{Seinfeld2016}.
Differences could be associated with weather conditions as cloudy days of September (4-5, 16-18, 26-30) of 2018, shown in Appendix (Figure~\ref{fig:rain_sep18}).

\begin{figure}[ht]
  \includegraphics[width=1\textwidth]{fig/Sep18_type_subplot_o3.pdf}
  \caption{Comparison between modeled and simulated values for surface ozone during September 2018 period considering all stations classified by type.}
  \label{fig:Sep18_type_o3}
\end{figure}

\begin{figure}[!hb]
  \begin{center}
    \includegraphics[width=.7\textwidth]{fig/MDA8_type.pdf}
  \end{center}
  \caption{Comparison between modeled and simulated values based on MDA8 surface ozone during September 2018 period for station types.}
  {\scriptsize Note. Daily maximum 8 h rolling mean (MDA8) of surface ozone concentrations.}
  \label{fig: MDA8_type_current}
\end{figure}

\subsubsection{Other pollutants related to ozone formation} 
Charts by station type for remaining pollutants (NO$_x$, CO, and Toluene) are shown in Appendix~\ref{ap:res}.
Pinheiros and S.André-Capuava stations measured toluene and the comparison against WRF-Chem simulation are presented in Figure~\ref{fig:tol}.
Figure~\ref{fig: Variation_pol_day} shows the primary (CO, NO$_x$) and secondary (O$_3$) pollutants diurnal profile.
Maximum ozone concentrations were reached between 12:00-16:00 hours, with peak concentration at 13:00 hours in average.
However, this does not correspond with observations when peak concentration is reached between 14:00-15:00 hours, as shown in Figure~\ref{fig:aqHour}.
One of the ozone precursors (NO$_x$) builds up during the morning rush hour and ending afternoon time, associated with traffic rush hours in the MASP.
Photochemical activity reduces NO$_2$ concentrations and enhances ozone formation.
This behavior depends on the VOC/NO$_x$ regime.
In the MASP, a VOC-sensitive (NO$_x$-saturated) predominates.
Furthermore, flex-fuel vehicles can burn hydrous ethanol and gasohol.
They contribute with aldehydes \citep{Nogueira2014} that at high concentrations can lead to an increase in troposphere reactivity as a driver in the ozone formation.

\begin{figure}[!hbt]
	\begin{center}
		\includegraphics[width=0.7\textwidth]{fig/Variation_pol_day}
	\end{center}
  \caption{Average hourly concentration of model simulations during the course of a day (Sep. 2018) of some important pollutants}
  {\scriptsize Note.\\ Shaded area is the standard deviation.}
  \label{fig: Variation_pol_day}
\end{figure}

Model simulations for Pinheiros station as an hourly mean concentration of the simulated period are compared with observations, shown in Figure~\ref{fig:Var_pinh_day}.
These results extend our knowledge about the influence of hydrocarbons in photochemical activity.
Toluene modeled values appear to be over-predicted in night-time hours and underpredicted during daylight hours.
This hourly variation during the daylight time is an indicator of remain hydrocarbons and their contributions to ozone formation.
The statistic evaluation for toluene shows only no significant linear correlation for the S. André-Capuava station (Table~\ref{tab:stats_tol}).
Pinheiros station and others (Paulínia, SJC, and SJC-VV) have significant linear correlations.

\begin{figure}[ht]
  \centering
  \includegraphics[width=.7\textwidth]{fig/pol_hour_tol.pdf}
  \caption{Average hourly concentrations comparison between observed (Obs.) and modeled (Mod.) values during the course of a day.}
  \label{fig:Var_pinh_day}
\end{figure}

There are several possible explanations for this result.
Temporal distribution of road transport emission may likely has contributed to inaccuracies because it represents the year 2014 \citep{Andrade2015}.
Another source of inaccuracy is related to vehicle fleet averaged by month, which could not represent daily variations along the months, such as the weekend effect.
Further data collection, such as vehicle fleet by day and hour, would be needed to determine how temporal distribution and road transport emissions affect toluene and ozone simulations.
Thus, the Vein model \citep{Ibarra2018} could improve this results based on high spatial and temporal resolution of the vehicle fleet from real-time GPS.

\begin{table}
\centering
\caption{Statistical results for toluene in Sep-Oct 2018 period}
\label{tab:stats_tol}
\begin{tabular}{llllll}
\toprule
{} & Paulínia & Pinheiros & S.André-Capuava & S.José Campos & S.José Campos-Vista Verde \\
Statistic   &          &           &                 &               &                           \\
\midrule
n           &     1153 &      1441 &            1384 &          1455 &                      1457 \\
MB          &      0.7 &      3.65 &            3.14 &          2.31 &                      -0.1 \\
MAGE        &     3.85 &      6.68 &            5.03 &          3.25 &                      3.34 \\
RMSE        &     6.52 &      9.36 &            7.02 &          4.89 &                       5.1 \\
NMB         &    17.36 &     60.26 &           74.84 &         196.9 &                     -2.47 \\
NME         &    96.04 &    110.11 &          119.77 &        276.35 &                      81.5 \\
IOA         &     0.39 &      0.42 &            0.31 &          0.33 &                      0.51 \\
r           &     0.19 &      0.15 &            0.03 &          0.13 &                      0.26 \\
Mm          &      4.7 &      9.72 &            7.34 &          3.49 &                       4.0 \\
Om          &     4.01 &      6.06 &             4.2 &          1.17 &                       4.1 \\
Msd         &     3.47 &      6.98 &            5.28 &          3.08 &                      3.53 \\
Osd         &     6.16 &      6.22 &            3.55 &          3.45 &                      4.73 \\
t-statistic &     6.57 &      5.76 &            1.12 &           5.0 &                     10.27 \\
t critical  &     1.96 &      1.96 &            1.96 &          1.96 &                      1.96 \\
Significant &     True &      True &           False &          True &                      True \\
\bottomrule
\end{tabular}
\end{table}



\section{Future changes under RCP 4.5 and 8.5 scenarios} \label{sec:res_fut}
% Third specific objective: Obtain surface ozone concentrations from the WRF-Chem model based on the RCP scenarios and compared them with the model results representative of the control-case study.
Model results were obtained for Sep-Oct 2030 period, considering two RCP emission scenarios as meteorology IC/LBC from the CESM Bias-Corrected dataset. We used the same emissions files in September-October 2018, therefore any variation on O$_3$ concentration is caused by the projections of the meteorological conditions.

The RCP 4.5 is a stabilization scenario that represents an emission mitigation through changes in the energy system, including shifts to electricity from lower emissions energy technologies, carbon capture and geologic storage technology \citep{Thomson2011}.
The worst-case scenario analyzed is the RCP 8.5.
It represents a very high baseline emission scenario, based on a non-climate policy known as "business as usual", combined with growing high population and high demands for fossil fuel and food \citep{Riahi2011}.
These two scenarios were compared with simulations values from "current" conditions (Sep-Oct 2018).

\subsection{Changes in meteorological conditions}\label{subsec:res_chan_met}
Model simulations show different temperature variations for September and October.
As shown in Appendix (Figure~\ref{fig:tc_change}), the RCP~8.5 scenario presented higher temperature values than RCP~4.5 and than current (Sep. 2018) scenarios.
Daily mean temperatures, based on the RCP~8.5, reach 32$^\circ$C at the end of September and in the first week of October 2030.
However, temperature simulations for October 2030 under two RCP scenarios presented similar variation ranges as in October 2018.

For monthly average, we found higher temperatures in September 2030 based on the RCP~8.5 scenario as shown in Figure~\ref{fig:temp_scen}.
The rising temperature affects the biogenic emissions inside the WRF-Chem because they are dependent on the temperature.
The most intriguing result arises from the comparison of monthly mean temperatures between the current 2018 and the RCP~4.5 scenarios.
It is interesting to note that temperatures for Sep. 2018 are very similar to the RCP~4.5 scenario for Sep. 2030 (getting ahead 12 years in the future!).
Moreover, October's temperature simulations were similar, where monthly mean values for Oct. 2018 are slightly above the RCP~4.5 scenario but below the RCP~8.5.
These results suggest that the RCP scenarios could underestimate the future rising temperatures for the S\~{a}o Paulo state, which could be higher.

Daily mean relative humidity variations are shown in Figure~\ref{fig:rh_change} for current and future scenarios.
These results are inversely proportional to the temperature values as shown in Figure~\ref{fig:rh_scen}.
Low values of relative humidity could occur for the RCP~8.5 scenario, mainly for September 2030.
In October, the current scenario presented high relative humidity values than the RCP 4.5 and 8.5 scenarios.

Wind speed and direction did not show marked differences between current and future RCP scenarios (Figure~\ref{fig:ws_change}).
Regarding accumulated rain (Figure~\ref{fig:rain_change}), the RCP~4.5 scenario presented higher daily modeled values between September 20-24 (2030).
However, total monthly rain values reveal a decreasing for the RCP scenarios (Figure~\ref{fig:rain_change_iag}), in which the RCP~8.5 could present low values of total monthly rain in the future (2030).

 \begin{figure}[hbt]
  \includegraphics[width=1\textwidth]{fig/temp_sep_oct.pdf}
  \caption{Monthly mean temperature by scenario and station: (a) September, (b) October.}
  \label{fig:temp_scen}
\end{figure}

 \begin{figure}[hbt]
  \includegraphics[width=1\textwidth]{fig/rh_sep_oct.pdf}
  \caption{Monthly mean relative humidity by scenario and station: (a) September, (b) October.}
  \label{fig:rh_scen}
\end{figure}

\begin{figure}[hbt]
  \centering
  \includegraphics{fig/rain_change_iag.pdf}
  \includegraphics{fig/rain_bymonth.pdf}
  \caption{Daily and monthly total rain by scenarios for the IAG/USP weather station}
  \label{fig:rain_change_iag}
\end{figure}

\subsection{Changes in surface ozone}\label{subsec:res_chan_o3}
The changes of surface O$_3$ concentrations from the 2018 to 2030 were analyzed by month due to different weather conditions.

\subsubsection{September}
Figure~\ref{fig:o3_changes} shows the comparison among simulated surface ozone concentrations for the three scenarios: Current (Sep. 2018), RCP~4.5 (Sep. 2030), and RCP~8.5 (Sep. 2030).
From the chart, we can observe that there are peak concentrations for some days (Sep. 8-11, 22-24) of the current scenario greater than in future conditions.
The industry type station's model simulations presented peak concentrations associated mainly with the RCP~4.5 scenario greater than the RCP~8.5 scenario.
For lasts days of September, the RCP~4.5 scenario also presented peak values in the MASP stations.
There are many days with peak concentrations associated with the RCP~8.5 scenario, mainly between Sep. 25-29 in the MASP.

\begin{figure}[!hbt]
\begin{center}
  \includegraphics[width=1.05\textwidth]{fig/rcp_2030_subplot_o3}
\end{center}
  \caption{Surface ozone concentrations for Sep. 2018 compared with RCP 4.5 and 8.5 for Sep. 2030.}
  \label{fig:o3_changes}
\end{figure}

\begin{figure}[!hbt]
\begin{center}
	\includegraphics{fig/MDA8_type_rcps}
\end{center}
  \caption{Maximum daily 8-hr rolling mean for surface ozone concentrations for Sep 2018 and 2030 (RCP 4.5 and 8.5)}
  \label{fig:MDA8_rcps}
\end{figure}

On average, the most remarkable result to emerge from the model simulations is that the RCP 8.5 scenario presented higher peak concentrations in the MASP, represented by station types as Urban, Urban park, and Forest preservation.
Outside the MASP (Regional urban, and Industry station types), the model simulations did not reveal surface ozone increases for future conditions based on the RCP 4.5 and 8.5 scenarios.   

Further analyses carried out for the maximum daily 8-hr rolling mean (MDA8) for ozone concentrations confirmed the initial findings regarding MDA8 increases in urban areas inside the MASP.
Figure~\ref{fig:MDA8_rcps} reveals many days with higher MDA8 ozone concentrations in the MASP, based on the RCP~8.5 scenario as meteorology IC/BC.
However, there are few days with higher concentrations belonging to the current scenario (Sep. 2018) and to the RCP~4.5 at the beginning and the last days of September.

Based on MDA8 ozone results, Figure~\ref{fig:spatial_o3_sep} shows spatial changes of surface ozone concentration as monthly mean difference between Sep. 2030 (RCP 4.5 and 8.5) and Sep. 2018.
Simulations for the RCP~4.5 show decrease surface O$_3$ concentrations in the MASP, whereas the RCP~8.5 simulations presented increase, mainly in urban areas.
This result confirms previous findings in the time series analysis in which the RCP~8.5 scenario presented many days with higher peak concentrations due to the increase in temperature.
Regarding changes of surface ozone for the RCP~4.5, decreasing concentrations are also confirmed in previous findings by \citet{Schuch2020}, considering results for 2030 under two emission scenarios (i.e., Current Legislation, Mitigation, and Maximum Feasible Reduction), with anthropogenic emissions variation depending on the chosen scenario.

\subsubsection{October}
Model simulations for October are different with higher peak concentrations when compared with the September simulation results shown in the previous section.
Figure~\ref{fig:o3_rcp_oct} shows a clear trend in the increasing of surface ozone concentrations for both RCP scenarios in 2030 compared with October 2018.
Although the RCP~8.5 is the worst-case scenario, we found much higher simulated surface ozone values for the RCP~4.5 than for the RCP~8.5, markedly in days between Oct. 20 to 25.
The most remarkable finding from the data analysis is that in urban stations, occurred a higher peak concentration with almost 320 $\mu$g~m$^{-3}$ corresponded to the RCP~4.5 scenario.
Only some days for the RCP~8.5 presented higher concentrations than the other scenarios (i.e., current conditions and the RCP~4.5 for Oct. 2018).

\begin{figure}[!hbt]
  \includegraphics[width=1\textwidth]{fig/rcp_2030_oct_subplot_o3}
  \caption{Surface ozone concentrations for Oct. 2018 compared with RCP 4.5 and 8.5 in Oct. 2030.}
  \label{fig:o3_rcp_oct}
\end{figure}

Model results of MDA8 ozone concentrations, shown in Figure~\ref{fig:MDA8_rcp_oct}, revealed some days with much higher concentrations for the RCP~8.5 and 4.5 scenarios.
However, the RCP~8.5 scenario presented for some days (20-25 Oct.) low values of MDA8 ozone concentrations than the current and the RCP~4.5 scenarios.
In the last days of October, MDA8 ozone concentrations for the RCP~8.5 present lower values than the October 2018 period.
These last days present rainy conditions for the RCP~8.5 scenario. 

\begin{figure}[!hbt]
  \begin{center}
	\includegraphics{fig/MDA8_type_rcps_oct18}
  \end{center}
  \caption{Maximum daily rolling mean for surface ozone concentrations for Oct 2018 and 2030 (RCP 4.5 and 8.5)}
  \label{fig:MDA8_rcp_oct}
\end{figure}

These results thus need to be interpreted with caution.
A possible explanation for this result is that cloudy days and low temperatures influenced differently each scenario.
Changes in MDA8 surface ozone concentrations were identified in Figure~\ref{fig:spatial_o3_oct}. 
From the chart, we can note that the RCP~4.5 presented more increases in the MDA8 ozone concentrations than the RCP~8.5.
Some locations for the RCP~8.5 presented decreases in ozone concentrations, especially when they are far away from the urban center of the MASP.
On average, we found that in the two RCP scenarios can be observed increase in the surface ozone concentration in the urban area of the MASP.

% List of Figures for this Chapter

\begin{figure}[hbt]
\begin{center}
	\includegraphics[width=1\textwidth]{fig/MDA8_spatial_station}
\end{center}
  \caption{MDA8 ozone mean spatial changes between Sep 2030 (RCP 4.5 and 8.5) and Sep. 2018}
  \label{fig:spatial_o3_sep}
\end{figure}

\begin{figure}[hbt]
  \includegraphics[width=1\textwidth]{fig/MDA8_spatial_station_oct}
  \caption{MDA8 ozone mean spatial changes between Oct. 2030 (RCP 4.5 and 8.5) and Oct. 2018}
  \label{fig:spatial_o3_oct}
\end{figure}



   

