\chapter{WRF-Chem files}
%\input{appen/emissions}
\section{Namelist used to run the WRF-Chem model}\label{ap03}
\subsection{NCEP FNL and RCP namelist.wps}
WPS stage in WRF-Chem modeling requires a \verb|namelist.wps| to prepare information related to geographical and meteorological data for a limited area for initial and lateral-boundary conditions.
For current (Sep-Oct 2018) simulation, WPS processing required the follow namelist:

	\begin{singlespace*}
		{\scriptsize \input{appen/namelist.wps_NCEP_Fin}}
	\end{singlespace*}
For future simulation (Sep-Oct 2030), the namelist used is shown bellow.
The only difference with the previous namelist is the line "fg\_name" source name.
For instance, \verb|CCSM4_CMIP5_MOAR_BC_RCP45| represents global projections for the RCP 4.5 emissions scenario.
\newpage
\begin{singlespace*}
	{\scriptsize \input{appen/namelist.wps_RCP}}
\end{singlespace*}


\subsection{NCEP FNL and RCP namelist.input}
The first simulation period with two days of spin-up, the namelist.input changed in:

\begin{singlespace*}
{\scriptsize
	\begin{verbatim}
	&time_control
	io_form_auxinput12 = 0

	&chem
	chem_in_opt        = 0,    0, 
	\end{verbatim}
}
\end{singlespace*}

After that, the namelist.input for the next simulation period requires chemical initial conditions to continue the simulation, where \verb|wrf_chem_input| must be linked with the wrfout first hours results of the period simulation.
For instance, the follow namelist.input started in September 5 00:00 h, so the wrfout results of the first simulation must be correspond to that hour and linked to \verb|wrf_chem_input|.
 
	\begin{singlespace*}
		{\scriptsize \input{appen/namelist.input_NCEP_Fin}}
	\end{singlespace*}

\section {Anthropogenic Emissions Calculation}\label{ap02}

\subsection{EDGAR-HTAP processing for industry and residential sectors}
EDGAR-HTAP has monthly emissions datasets (yearly and monthly files) for different sectors (e.g., industry, residential), available in its \href{https://edgar.jrc.ec.europa.eu/htap_v2/}{website}.
According to it, datasets contain 0.1 $\times$ 0.1 deg gridmaps of CH$_4$, CO, SO$_2$, NO$_x$, NMVOC, NH$_3$, PM$_{10}$, PM$_{2.5}$, BC and OC for the years 2008 and 2010.
\begin{figure}[!htb]
	\centering
  \includegraphics[width=.7\textwidth]{fig/voc_frac.pdf}
  \caption{VOC fractions}
  \label{fig: voc_frac}
\end{figure}
These datasets "use nationally reported emissions combined with regional scientific inventories in the format of sector-specific gridmaps. The gridmaps are complemented with EDGARv4.3 data for those regions where data are absent."
However, these datasets don't contain specific emissions about NMVOC speciation (e.g., ethane, toluene, aldehydes, formaldehydes, etc.) needed for the CBM-Z mechanism in the WRF-Chem for photochemical reactions (i.e., ozone formation).
 Before running the \verb|anthro_emiss| for each sector, the NMVOC emission files were converted to different VOC species based on relative fractions of VOC emissions from different processes and fuels (Figure 6 in \citealt{Andrade2015}). 
Procedures to obtain speciation emission files from NMVOC are as follow:

\begin{itemize}
	\item \verb|voc_frac.py| uses information from LAPAt emission preprocessor for road transport (\verb|wrfchemi_00z_d01_veic|) to generate a  \verb|voc_frac_round.csv| (Figure~\ref{fig: voc_frac}).
	\item \verb|htap2AE_year.py| uses the \verb|voc_frac_round.csv| file and NMVOC emission files from EDGAR-HTAP to generate VOC speciation emission files.
\end{itemize}

WRF-Chem emission files (\verb|wrfchemi_<hh>z_<domain>|) for industry and residential are generated through running ANTHRO EMISS in the Linux terminal:\\
\verb|./anthro_emiss < cbmz.inp > cbmz.out| \\
For instance, the cbmz.inp namelist contents the follow information for the industry sector:

{\scriptsize
\begin{verbatim}
	&CONTROL
 anthro_dir = '/scr2/alejandro/WRF/DATA/util/EDGAR-HTAP/industry'
 wrf_dir    = '/scr2/alejandro/WRF/SEP18/WRF'
 domains    = 2
 src_file_prefix = 'edgar_HTAP_'
 src_file_suffix = '_emi_INDUSTRY_2010.0.1x0.1_AE.nc'
 src_names = 'CO(28)','NOx(30)','SO2(64)','NH3(17)','ISO(68.12)','ETH(30.07)','HC3(42.66)',
             'HC5(60)','HC8(96)','XYL(104)','OL2(28.05)','OLT(56)','OLI(56)','TOL(92)',
             'HCHO(30.09)','ALD(44.05)','KET(58.08)','CH3OH(70.09)','C2H5OH(46)'
 sub_categories  = 'emis_tot'
 cat_var_prefix  = ' '
 serial_output   = .false.
 start_output_time  = '2018-09-29_00:00:00'
 stop_output_time  = '2018-11-01_00:00:00'
 output_interval = 3600
 data_yrs_offset = 8
 emissions_zdim_stag = 1
 emis_map = 'CO->CO','NO->0.9*NOx','NO2->0.1*NOx','SO2->SO2','NH3->NH3',
            'ISO->ISO','ETH->ETH','HC3->HC3','HC5->HC5','HC8->HC8',
            'XYL->XYL','OL2->OL2','OLT->OLT','OLI->OLI','TOL->TOL',
            'HCHO->HCHO','ALD->ALD','KET->KET','CH3OH->CH3OH','C2H5OH->C2H5OH'
\end{verbatim}}

Python scripts (\verb|voc_frac.py| and \verb|htap2AE_year.py|) can get on this \href{https://github.com/adelgadop/Master_Dissertation}{GitHub}.

\subsection{LAPAt emission preprocessor for road transport}
LAPAt preprocessor emission model used in \citet{Andrade2015} contents a namelist (\verb|namelist_fc.emi|) and a NCL script (\verb|wrfchemi_cbmz_fc.ncl|) developed by researchers of IAG as a utility for generating ready emission files for WRF-Chem.
The NCL script (\verb|wrfchemi_cbmz_fc.ncl|) calculates road transport emission based through the Bottom-Up methodology based on data files called in the \verb|namelist_fc.emi|.
For instance, the follow \verb|namelist_fc.emi| applies to the parent domain with 15 km $\times$ 15 km for September 2018:

{\scriptsize \input{appen/namelist_fc.emi_d01}}
LAPAt pre-processor emissions model created WRF-Chem ready emissions files for two modeling domains, considering the following structure: \verb|wrfchemi_{00z, 12z}_{d01, d02}_veic|, using the follow code in the Linux terminal of the Master IAG server (svan2): \\
		\verb|ncl wrfchemi_cbmz_fc.ncl|\\
The total number of vehicle approach for each modeling domain area was based on the spatial data sets of Brazil (available in \href{https://github.com/ipeaGIT/geobr}{geobr} Python package) and information about vehicle types by each municipality downloaded from \href{https://www.gov.br/infraestrutura/pt-br/assuntos/transito/conteudo-denatran/frota-de-veiculos-2018}{DENATRAN}.
A Python script `01\_Vehicles.py' was used to calculate the total number of vehicles and fraction by type and fuel consumption, available in this \href{https://github.com/adelgadop/Master_Dissertation}{GitHub}.

The primary data files are \verb|grid15km_d01.txt| and \verb|grid03km_d02.txt|, which are the results from QGIS and Python processing, shown below (next page).
These datasets represent the total road length as the sum for each grid cell based on the motorway, trunk, primary, secondary, and tertiary types.
\includepdf[pages= -]{appen/Tutorial_English_Rev_1.pdf}

\subsection{Biogenic emissions}\label{ap: biogenic}
The \verb|megan_bio_emiss| utility reads transformed MEGAN biogenic input files and create \verb|wrfbiochemi_d<nn>| files needed to run the WRF-Chem model.
	All the executables can be created with the \verb|make_util| script.
	MEGAN version 2 can be run using the follow linecode in the Linux terminal: \\
	\verb|./megan_bio_emiss < megan_bio_emiss.inp|\\
	The \verb|megan_bio_emiss.inp| file for this study contents the follow information and cover September and October:
	
	\begin{verbatim}
	
		&control
		domains = 2,
		start_lai_mnth = 7,
		end_lai_mnth = 11,
		wrf_dir = '/scr2/alejandro/WRF/Y2018/WRF'
		megan_dir = '/scr2/alejandro/WRF/DATA/util/MEGAN/data'
		/
	\end{verbatim}
	
	Two biogenic emission files were created (\verb|wrfbiochemi_d01| and \verb|wrfbiochemi_d02|).
	For instance, Figure~\ref{fig:bioemis} shows information for the parent domain (\verb|wrfbiochemi_d01|). 
	
	\begin{figure}[!htb]
		\includegraphics[width=1\textwidth]{fig/biogenic_d01.pdf}
  		\caption{Spatial distribution of MEGAN version 2 in the 15 km parent modeling domain for September.}
  		\centering
  		\label{fig:bioemis}
	\end{figure}
	
\chapter{Air Quality and Meteorological Information}\label{ap01}
There are several air quality and meteorological stations managed by CETESB for the S\~{a}o Paulo state.
The parameters were downloaded from \href{https://qualar.cetesb.sp.gov.br/}{QUALAR} or the next url: \url{https://qualar.cetesb.sp.gov.br/} and correspond to:

\begin{itemize}
	\item Meteorological parameters
		\begin{itemize}
			\item Temperature at 2 m above ground [$^o$C]
			\item Relative humidity at 2 m above ground [\%]
			\item Solar radiation [W/m$^2$]
			\item Wind speed at 10 m above ground [m/s]
			\item Wind direction at 10 m above ground [degrees]
		\end{itemize}
		
	\item Air quality parameters
		\begin{itemize}
			\item Surface ozone concentration [$\mu$g m$^{-3}$]
			\item Nitrogen monoxide concentration [$\mu$g m$^{-3}$]
			\item Nitrogen dioxide concentration [$\mu$g m$^{-3}$]
			\item Carbon monoxide concentration [ppm] 
		\end{itemize}
\end{itemize}

For September and October, air quality and meteorological parameters were downloaded from CETESB, considering all stations with hourly data available for five years (2014-2018).
For other months, only ten stations were downloaded in order to recognize and justify which month recorded high surface ozone concentrations.
Figure~\ref{fig:Data} shows monthly distribution of hourly data by station type, and covers five years as time series.
These stations represent different land use types and they were named in this study as "Industry", "Regional urban", "Urban park", "Urban", and "Forest preservation".
Not all station types are inside the MASP (i.e., Regional urban and Industry). 
 Hourly time series of weather and air quality parameters were automatically downloaded using a Python scripts, named \verb|downloaded_CETESB.py| and \verb|qualar_py.py| (developed by \href{https://github.com/quishqa/qualR.py}{M. Gavidia}).
The repository of these scripts used in this study are available in this \href{https://github.com/adelgadop/Master_dissertation/tree/main/02_Obs_scripts}{GitHub link}.

	\begin{figure}[!ht]
		\centering
 		\includegraphics[width=0.85\textwidth]{fig/Air_Met_boxplot.pdf}
  		\caption{Air quality and meteorological hourly data, downloaded from CETESB (QUALAR)}
  		\label{fig:Data}
	\end{figure}
	
	\begin{figure}[hbt]
	\begin{center}
		\includegraphics[width=0.9\textwidth]{fig/HC_CETESB_sep_oct2018.pdf}
	\end{center}
  		\caption{Hydrocarbon measurements in some CETESB stations as Toluene and Benzene.}
  		\label{fig:HC_obs}
	\end{figure}


Daily maximum rolling 8-hour mean (MDA8) for ozone concentration were compared with air temperature daily mean considering five years and by station type as shown in Figure~\ref{fig:o3_years}.
Temperature has a positive correlation with surface ozone when the weather is not cloudy as occurred during highest rainfall in the MASP from November to March \citep{Lima2018}.
When solar radiation is not reduced by cloud cover, months between September and December presented high hourly mean ozone concentration as shown in Figure~\ref{fig:o3_time}.
September and October stand out as the months with the highest concentrations of ozone.

  \begin{table}[!ht]
  	\centering
	\caption{Air quality and weather stations network in the MASP}
	\label{tab:masp_stations}
  	\begin{tabular}{lrrll}
	\toprule
                       Name &   Latitude &  Longitude &                 Type &    Abb \\
	\midrule
                        IAG & -23.651200 & -46.622400 &  Forest preservation &    IAG \\
            Pico do Jaraguá & -23.456269 & -46.766098 &  Forest preservation &   PdJr \\
                Santo Amaro & -23.654977 & -46.709998 &                Urban &    SAm \\
            S.André-Capuava & -23.639804 & -46.491637 &                Urban &   SACp \\
             S.André-Centro & -23.645616 & -46.536335 &                Urban &   SACt \\
     S.André-Paço Municipal & -23.656994 & -46.530919 &                Urban &   SAPM \\
          S.Bernardo-Centro & -23.698671 & -46.546232 &                Urban &    SBC \\
       S.Bernardo-Paulicéia & -23.671354 & -46.584668 &                Urban &    SBP \\
         Grajaú-Parelheiros & -23.776266 & -46.696961 &                Urban &    GrP \\
    Marg.Tietê-Pte Remédios & -23.518706 & -46.743320 &                Urban &   MTPR \\
         Guarulhos-Pimentas & -23.440117 & -46.409949 &                Urban &   GrlP \\
                    Cambuci & -23.567708 & -46.612273 &                Urban &    Cmb \\
          S.Miguel Paulista & -23.498526 & -46.444803 &                Urban &   SMPa \\
             Itaim Paulista & -23.501547 & -46.420737 &                Urban &   ItPa \\
            Taboão da Serra & -23.609324 & -46.758294 &                Urban &    TdS \\
                 Interlagos & -23.680508 & -46.675043 &                Urban &   Intr \\
                  Guarulhos & -23.463209 & -46.496214 &                Urban &   Grlh \\
         São Caetano do Sul & -23.618443 & -46.556354 &                Urban &   SCdS \\
                    Santana & -23.505993 & -46.628960 &                Urban &   Stna \\
                Carapicuíba & -23.531395 & -46.835780 &                Urban &    Crp \\
                    Diadema & -23.685876 & -46.611622 &                Urban &    Dia \\
                       Mauá & -23.668549 & -46.466000 &                Urban &   Maua \\
            Mogi das Cruzes & -23.518172 & -46.186861 &                Urban &    MdC \\
                      Mooca & -23.549734 & -46.600417 &                Urban &    Mca \\
             N.Senhora do Ó & -23.480099 & -46.692052 &                Urban &   NSdO \\
          Parque D.Pedro II & -23.544846 & -46.627676 &                Urban &  PDPII \\
                     Osasco & -23.526721 & -46.792078 &                Urban &   Ossc \\
            Cerqueira César & -23.553543 & -46.672705 &                Urban &    CeC \\
                  Pinheiros & -23.561460 & -46.702017 &                Urban &   Pinh \\
                     Centro & -23.547806 & -46.642415 &                Urban &   Ctro \\
   Guarulhos-Paço Municipal & -23.455534 & -46.518533 &                Urban &    GPM \\
                  Congonhas & -23.616320 & -46.663466 &                Urban &    Cng \\
              Capão Redondo & -23.668356 & -46.780043 &           Urban park &    CRe \\
 Cid.Universitária-USP-Ipen & -23.566342 & -46.737414 &           Urban park &    USP \\
                   Itaquera & -23.580015 & -46.466651 &           Urban park &   Itqr \\
                 Ibirapuera & -23.591842 & -46.660688 &           Urban park &   Ibir \\
	\bottomrule
	Notes:\\
	IAG station belongs to USP.\\
	Abb = abbreviation.
	\end{tabular}
\end{table}

\begin{table}
  	\centering
	\caption{Air quality and weather stations network in surrounding areas the MASP in the São Paulo State}
	\label{tab:sp_stations}
	\begin{tabular}{lrrll}
	\toprule
                       Name &   Latitude &  Longitude &                 Type &    Abb \\
	\midrule
            Santa Gertrudes & -22.459955 & -47.536298 &             Industry &  StGrt \\
                   Paulínia & -22.772321 & -47.154843 &             Industry &    Pln \\
                    Jundiaí & -23.192004 & -46.897097 &       Regional urban &    Jnd \\
                    Limeira & -22.563604 & -47.414314 &       Regional urban &    Lmr \\
                    Taubaté & -23.032351 & -45.575805 &       Regional urban &    Tbt \\
               Paulínia Sul & -22.786806 & -47.136559 &       Regional urban &   PlnS \\
                 Piracicaba & -22.701222 & -47.649653 &       Regional urban &   Prcb \\
            Pirassununga-EM & -22.007713 & -47.427564 &       Regional urban &   PrEM \\
        Presidente Prudente & -22.119937 & -51.408777 &       Regional urban &   PrPr \\
             Ribeirão Preto & -21.153942 & -47.828481 &       Regional urban &   RbPr \\
      São José do Rio Preto & -20.784689 & -49.398278 &       Regional urban &   SJRP \\
              S.José Campos & -23.187887 & -45.871198 &       Regional urban &   SJCp \\
  S.José Campos-Jd.Satelite & -23.223645 & -45.890800 &       Regional urban &   SJCJ \\
  S.José Campos-Vista Verde & -23.183697 & -45.830897 &       Regional urban &   SJCV \\
                   Sorocaba & -23.502427 & -47.479030 &       Regional urban &   Srcb \\
                      Tatuí & -23.360752 & -47.870799 &       Regional urban &     Tt \\
                        Jaú & -22.298620 & -48.567457 &       Regional urban &    Jau \\
                    Jacareí & -23.294199 & -45.968234 &       Regional urban &    Jcr \\
                  Americana & -22.724253 & -47.339549 &       Regional urban &    Ame \\
                  Catanduva & -21.141943 & -48.983075 &       Regional urban &    Cnt \\
                  Araçatuba & -21.186841 & -50.439317 &       Regional urban &    Ara \\
                 Araraquara & -21.782522 & -48.185832 &       Regional urban &   Arrq \\
                      Bauru & -22.326608 & -49.092759 &       Regional urban &    Bau \\
            Campinas-Centro & -22.902525 & -47.057211 &       Regional urban &    CpC \\
          Campinas-Taquaral & -22.874619 & -47.058973 &       Regional urban &    CpT \\
           Campinas-V.União & -22.946728 & -47.119281 &       Regional urban &    CpV \\
                    Marília & -22.199809 & -49.959970 &       Regional urban &    Mrl \\
              Guaratinguetá & -22.801917 & -45.191122 &       Regional urban &   Grtg \\
	\bottomrule
	\end{tabular}
\end{table}

	\begin{figure}[htbp]
 		\includegraphics[width=1\textwidth]{fig/ozone_2014_2018_series.pdf}
  		\caption{Time series of maximum daily rolling 8-hr mean for ozone and daily mean for temperature}
   		{\scriptsize Note: \\ Station types as Forest preservation (Pico do Jaraguá), Urban (Interlagos, Carapicuíba, Parque D.Pedro II, Pinheiros), and Urban park (Ibirapuera, Itaquera) are inside the MASP.
  		Others as Industry (Paulínia) and Regional urban (Campinas-Taquaral, Sorocaba) are outside the MASP, and they belong to São Paulo state.}
  		\label{fig:o3_years}
	\end{figure}

	\begin{figure}[!htb]
  		\includegraphics[width=1\textwidth]{fig/NoNo2_ratio.pdf}
  		\caption{Hourly mean concentrations for NO and NO$_2$ comparison for September 2018 for station types in the MASP.}
  		\label{fig:NoNo2_ratio}
	\end{figure}
	
	\begin{figure}[htbp]
 		\includegraphics[width=1\textwidth]{fig/o3_hourly_type_obs.pdf}
  		\caption{Surface ozone variation by month and station type in São Paulo State}
  		{\scriptsize Note: \\ Station types as Forest preservation (Pico do Jaraguá), Urban (Interlagos, Carapicuíba, Parque D.Pedro II, Pinheiros), and Urban park (Ibirapuera, Itaquera) are inside the MASP.
  		Others as Industry (Paulínia) and Regional urban (Campinas-Taquaral, Sorocaba) are outside the MASP, and they belong to São Paulo state.} 
  		\label{fig:o3_time}
	\end{figure}
	
	\begin{figure}
  		\includegraphics[width=1\textwidth]{fig/IAG_met.pdf}
  		\caption{Hourly time series of meteorological parameters registered at IAG/USP weather station.}
  		{\footnotesize Note. Wind speed (ws), wind direction (wd), 2-m temperature (tc), relative humidity (rh), atmospheric pressure (press), rain rate (rr), sunlight duration (sun), cloud cover (cc).}
  		\label{fig:met_iag}
  	\end{figure}
  	
	\begin{figure}
  		\centering
  		\includegraphics[width=1\textwidth]{fig/rain_cc_IAG.pdf}
  		\caption{Total daily rain and cloud cover at IAG/USP weather station.}
  		\label{fig:rain_cc_iag}
  	\end{figure}
  	
  	\begin{figure}
  		\includegraphics[width=1\textwidth]{fig/rain_cc.pdf}
  		\caption{Total daily rain and cloud cover in September 2018 based on data collected at IAG/USP weather station}
  		{\scriptsize Note:\\ Gray highlight represents excluding days for the paired statistical evaluation between modeled and observed values.}
  		\label{fig:rain_sep18}
  	\end{figure}
		

%\includepdf[pages= -]{appen/namelist_input_NCEP_FNL.pdf}
%\includepdf[pages= -]{appen/namelist_input_RCPs.pdf}

 \chapter{Statistical metrics}\label{ap04}
 For this study, the statistical evaluation was developed using a Python script called \verb|mod_stats|.
  This script was based on the original script \verb|model_stats| developed by M. Gavidia's \href{https://github.com/quishqa/WRF-Chem_SP/tree/master/wrf_sp_eval}{GitHub}.
  Functions that are included in the script \verb|mod_stats| (available in this \href{https://github.com/adelgadop/Master_dissertation/tree/main/04_wrfchem_scripts}{GitHub}) are :
  
  \begin{itemize}
  	\item \verb|aq_stats| evaluates air quality paired values
  	\item \verb|met_stats| evaluates meteorological paired values
  	\item \verb|wind_dir_diff| calculates difference between wind directions based on its periodic property based on \citet{Reboredo2015}.
  	\item \verb|wind_dir_mb| calculates wind direction mean bias based on \citet{Reboredo2015}.
  	\item \verb|wind_dir_mage| calculates wind direction mean absolute error based on \citet{Reboredo2015}.
  	\item \verb|r_pearson_sign| calculates Pearson's \textit{r} significance based on t-test with a two-tail (non-directional). The p-value is calculated using \verb|scipy.stats| Python module with n-2 degrees of freedom.
  	\item \verb|r_pearson_confidence_interval| calculates Pearson's \textit{r} confidence intervals using two-tail t-test.
  \end{itemize}
 
  These functions (\verb|aq_stats|, \verb|met_stats|) calculate statistical parameters described below:
  
  \begin{itemize}
  	\item \textit{n} is the number of paired values considered (modeled and observed values)
  	\item \textit{MB} is the Mean Bias
  	\item \textit{MAGE} is the Mean Absolute Gross Error
  	\item \textit{RMSE} is the Root Mean Square Error, which the ideal value is 0.
  	\item \textit{NMB} is the Normalize Mean Bias and reports mean paired modeled and observation differences normalized by the mean observation. Positive value corresponds to overprediction; negative value corresponds to underprediction.
  	\item \textit{NME} is the Normalized Mean Error and reports mean paired modeled and observation differences normalized by the mean observation as a positive value.
  	\item \textit{IOA} is the Index Of Agreement, which the perfect value is 1.
  	\item \textit{r} is the correlation coefficient based on Numpy module in Python
  	\item \textit{Mm} is the mean of modeled values
  	\item \textit{Om} is the mean of observed values
  	\item \textit{Msd} is the standard deviation of modeled values
  	\item \textit{Osd} is the standard deviation of observed values
  \end{itemize}
  
  \chapter{Model results and Post-processing}\label{ap:res}
  This appendix shows results for the primary pollutants (NO$_x$, CO, and toluene), and meteorological parameters.
  WRF-Chem model outputs were processed through Python scripts (\verb|wrf_extract.py|, \verb|wrf_rain.py|, \verb|mod_stats.py|).
  These scripts extracted pollutants and meteorological parameters for each station location, and for statistical analysis.
  Full codes are available on this \href{https://github.com/adelgadop/Master_dissertation/tree/main/04_wrfchem_scripts}{GitHub}.
  
  
  \begin{table}
\centering
\caption{Statistical results for meteorological parameters for Sep-Oct 2018 by type}
\label{tab:stats_all_type}
\begin{tabular}{lrrrrrrrrrrl}
\toprule
{} &      n &     MB &   MAGE &   RMSE &   IOA &     r &     Mm &     Om &    Msd &    Osd &                 type \\
\midrule
tc &  13040 &   1.33 &   1.85 &   2.44 &  0.91 &  0.88 &  20.98 &  19.65 &   4.05 &   4.17 &                Urban \\
rh &  13031 &  -7.01 &  10.58 &  13.77 &  0.81 &  0.73 &  71.24 &  78.25 &  16.26 &  15.80 &                Urban \\
ws &  18748 &   1.36 &   1.65 &   2.02 &  0.49 &  0.34 &   3.35 &   1.98 &   1.49 &   1.04 &                Urban \\
wd &  17869 & -21.15 &  44.19 &      - &     - &     - &      - &      - &      - &      - &                Urban \\
tc &   1416 &   1.48 &   1.87 &   2.48 &  0.91 &  0.88 &  20.42 &  18.94 &   4.00 &   4.09 &           Urban park \\
rh &   1416 & -10.03 &  12.24 &  15.34 &  0.79 &  0.74 &  74.06 &  84.08 &  15.93 &  16.20 &           Urban park \\
ws &   1429 &   1.58 &   1.72 &   2.08 &  0.52 &  0.47 &   3.39 &   1.82 &   1.48 &   1.05 &           Urban park \\
wd &   1264 & -10.24 &  32.67 &      - &     - &     - &      - &      - &      - &      - &           Urban park \\
tc &  18465 &   1.26 &   1.98 &   2.59 &  0.92 &  0.88 &  22.91 &  21.65 &   4.32 &   4.80 &       Regional urban \\
rh &  17045 &  -7.44 &  11.46 &  14.72 &  0.84 &  0.77 &  65.96 &  73.40 &  17.92 &  19.09 &       Regional urban \\
ws &  19427 &   1.20 &   1.64 &   2.08 &  0.58 &  0.46 &   3.31 &   2.11 &   1.86 &   1.24 &       Regional urban \\
wd &  18641 & -25.68 &  58.98 &      - &     - &     - &      - &      - &      - &      - &       Regional urban \\
tc &   2890 &   0.77 &   1.76 &   2.34 &  0.94 &  0.90 &  23.60 &  22.83 &   4.24 &   5.05 &             Industry \\
rh &   2890 &  -5.16 &  11.26 &  14.85 &  0.87 &  0.80 &  65.01 &  70.17 &  19.74 &  22.96 &             Industry \\
ws &   2888 &   1.71 &   1.99 &   2.49 &  0.43 &  0.34 &   3.29 &   1.58 &   1.88 &   0.99 &             Industry \\
wd &   2538 & -13.24 &  50.95 &      - &     - &     - &      - &      - &      - &      - &             Industry \\
tc &   2916 &   1.90 &   2.26 &   2.88 &  0.88 &  0.86 &  19.87 &  17.97 &   4.09 &   4.15 &  Forest preservation \\
rh &   2916 &  -8.87 &  11.65 &  15.10 &  0.80 &  0.72 &  76.61 &  85.48 &  16.68 &  16.14 &  Forest preservation \\
ws &   2920 &   2.18 &   2.27 &   2.69 &  0.31 &  0.33 &   3.74 &   1.57 &   1.67 &   0.75 &  Forest preservation \\
wd &   2877 & -11.27 &  43.36 &      - &     - &     - &      - &      - &      - &      - &  Forest preservation \\
\bottomrule
\end{tabular}
\end{table}



\begin{figure}[hbt]
  \begin{center}
  	\includegraphics[width=.8\textwidth]{fig/rh_cetesb}
  \end{center}
  \caption{Model results for 2-m relative humidity (Sep-Oct 2018) compared with observations from CETESB and IAG stations by types.}
  {\scriptsize Note. Time series as daily mean. Shaded area is the standard deviation.}
  \label{fig:rh_cetesb}
\end{figure}

\begin{figure}[hbt]
  \begin{center}
  	\includegraphics[width=.8\textwidth]{fig/temp_cetesb}
  \end{center}
  \caption{Model results for 2-m temperature (Sep-Oct 2018) compared with observations from CETESB and IAG stations by types.}
  {\scriptsize Note. Time series as daily mean. Shaded area is the standard deviation.}
  \label{fig:temp_cetesb}
\end{figure}

\begin{figure}[hbt]
  \begin{center}
    \includegraphics[width=.8\textwidth]{fig/ws_cetesb}
  \end{center}
  \caption{Model results for wind speed (Sep-Oct 2018) compared with observations from CETESB and IAG stations by station types.}
  {\scriptsize Note. Time series as daily mean. Shaded area is the standard deviation.}
  \label{fig:ws_cetesb}
\end{figure}
  
\begin{figure}[hbt]
  \begin{center}
  	\includegraphics[width=1\textwidth]{fig/Sep_Oct18_station_subplot_nox}
  \end{center}
  \caption{Model results for NO$_x$ (Sep-Oct 2018) compared with hourly time series from CETESB measurements.}
  {\scriptsize Note. Representative station types for Forest preservation (Pico do Jaragu\'{a}), Urban (Carapicu\'{i}ba), Urban park (Ibirapuera), Industry (Paulínia), and Regional urban (Campinas-Taquaral).}
\end{figure}

\begin{figure}[hbt]
  \centering
  \includegraphics{fig/Sep_Oct18_type_subplot_co}
  \caption{Model results for CO (Sep-Oct 2018) compared with hourly time series from CETESB measurements by station types.}
\end{figure}

\begin{figure}[hbt]
  \centering
  \includegraphics{fig/tseries_tol.pdf}
  \caption{Model results for Toluene compared with CETESB measurements.}
  \label{fig:tol}
\end{figure}

\begin{figure}[hbt]
  \includegraphics[width=1\textwidth]{fig/tc_change.pdf}
  \caption{Daily mean of temperature by scenario and station type}
  {\scriptsize The shaded area is the standard deviation.}
  \label{fig:tc_change}
\end{figure}

\begin{figure}[hbt]
  \includegraphics[width=1\textwidth]{fig/rh_change.pdf}
  \caption{Daily mean of relative humidity by scenario and station type}
  {\scriptsize The shaded area is the standard deviation.}
  \label{fig:rh_change}
\end{figure}

\begin{figure}[hbt]
  \includegraphics[width=1\textwidth]{fig/ws_change.pdf}
  \caption{Daily mean of wind speed by scenario and station}
  \label{fig:ws_change}
\end{figure}


\begin{figure}[hbt]
  \centering
  \includegraphics{fig/rain_change_all.pdf}
  \caption{Daily total rain for scenarios and station type}
  \label{fig:rain_change}
\end{figure}





  
