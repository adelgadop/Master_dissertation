\begin{table}
\centering
\begin{threeparttable}
\caption{Statistical results for meteorological parameters for Sep-Oct 2018 (IAG/USP station)}
\label{tab:stats_iag}
\begin{tabular}{lrrrrr}
\toprule
Statistic\tnote{(a)} & 2-m Temp. ($^{\circ}$C) & 2-m RH (\%) & Rain rate (mm)  & W. Speed (m~s$^{-1}$) & W. Dir. ($^{\circ}$)  \\
\midrule
n      &  1461.00 &  1461.00 &  1461.00 &  1461.00 &  1461.00 \\
MB     &     1.56 &    -9.00 &     0.13 &     1.69 &     5.02 \\
MAGE   &     1.97 &    11.15 &     0.42 &     1.84 &    32.57 \\
RMSE   &     2.66 &    14.66 &     1.72 &     2.22 &        - \\
IOA    &     0.89 &     0.78 &     0.23 &     0.43 &        - \\
r      &     0.85 &     0.71 &     0.11 &     0.37 &        - \\
Mm     &    20.32 &    74.30 &     0.30 &     3.37 &        - \\
Om     &    18.76 &    83.30 &     0.18 &     1.69 &        - \\
Msd    &     3.96 &    15.68 &     1.20 &     1.50 &        - \\
Osd    &     3.89 &    14.46 &     1.35 &     0.91 &        - \\
t-stat &    61.63 &    38.51 &     4.23 &    15.21 &        - \\
t-crit &     1.96 &     1.96 &     1.96 &     1.96 &        - \\
\bottomrule
\end{tabular}
\begin{tablenotes}
{\scriptsize
	\item[(a)] MB = Mean bias, MAGE = Mean Absolute Gross Error, RMSE = Root Mean Square Error, Mm = Mean of modeled values, Om = Mean of observed values, Msd = Standard deviation of modeled values, and Osd = Standard deviation of observed values. Units depend on the meteorological parameter. Correlation coefficient (r) is in dimensionless units. Statistical parameters are t-test statistical (t-stat) and t critical (t-crit).}
\end{tablenotes}
\end{threeparttable}
\end{table}

